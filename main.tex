\documentclass[11pt]{article}

    \usepackage[breakable]{tcolorbox}
    \usepackage{parskip} % Stop auto-indenting (to mimic markdown behaviour)
    

    % Basic figure setup, for now with no caption control since it's done
    % automatically by Pandoc (which extracts ![](path) syntax from Markdown).
    \usepackage{graphicx}
    % Keep aspect ratio if custom image width or height is specified
    \setkeys{Gin}{keepaspectratio}
    % Maintain compatibility with old templates. Remove in nbconvert 6.0
    \let\Oldincludegraphics\includegraphics
    % Ensure that by default, figures have no caption (until we provide a
    % proper Figure object with a Caption API and a way to capture that
    % in the conversion process - todo).
    \usepackage{caption}
    \DeclareCaptionFormat{nocaption}{}
    \captionsetup{format=nocaption,aboveskip=0pt,belowskip=0pt}

    \usepackage{float}
    \floatplacement{figure}{H} % forces figures to be placed at the correct location
    \usepackage{xcolor} % Allow colors to be defined
    \usepackage{enumerate} % Needed for markdown enumerations to work
    \usepackage{geometry} % Used to adjust the document margins
    \usepackage{amsmath} % Equations
    \usepackage{amssymb} % Equations
    \usepackage{textcomp} % defines textquotesingle
    % Hack from http://tex.stackexchange.com/a/47451/13684:
    \AtBeginDocument{%
        \def\PYZsq{\textquotesingle}% Upright quotes in Pygmentized code
    }
    \usepackage{upquote} % Upright quotes for verbatim code
    \usepackage{eurosym} % defines \euro

    \usepackage{iftex}
    \ifPDFTeX
        \usepackage[T1]{fontenc}
        \IfFileExists{alphabeta.sty}{
              \usepackage{alphabeta}
          }{
              \usepackage[mathletters]{ucs}
              \usepackage[utf8x]{inputenc}
          }
    \else
        \usepackage{fontspec}
        \usepackage{unicode-math}
    \fi

    \usepackage{fancyvrb} % verbatim replacement that allows latex
    \usepackage{grffile} % extends the file name processing of package graphics
                         % to support a larger range
    \makeatletter % fix for old versions of grffile with XeLaTeX
    \@ifpackagelater{grffile}{2019/11/01}
    {
      % Do nothing on new versions
    }
    {
      \def\Gread@@xetex#1{%
        \IfFileExists{"\Gin@base".bb}%
        {\Gread@eps{\Gin@base.bb}}%
        {\Gread@@xetex@aux#1}%
      }
    }
    \makeatother
    \usepackage[Export]{adjustbox} % Used to constrain images to a maximum size
    \adjustboxset{max size={0.9\linewidth}{0.9\paperheight}}

    % The hyperref package gives us a pdf with properly built
    % internal navigation ('pdf bookmarks' for the table of contents,
    % internal cross-reference links, web links for URLs, etc.)
    \usepackage{hyperref}
    % The default LaTeX title has an obnoxious amount of whitespace. By default,
    % titling removes some of it. It also provides customization options.
    \usepackage{titling}
    \usepackage{longtable} % longtable support required by pandoc >1.10
    \usepackage{booktabs}  % table support for pandoc > 1.12.2
    \usepackage{array}     % table support for pandoc >= 2.11.3
    \usepackage{calc}      % table minipage width calculation for pandoc >= 2.11.1
    \usepackage[inline]{enumitem} % IRkernel/repr support (it uses the enumerate* environment)
    \usepackage[normalem]{ulem} % ulem is needed to support strikethroughs (\sout)
                                % normalem makes italics be italics, not underlines
    \usepackage{soul}      % strikethrough (\st) support for pandoc >= 3.0.0
    \usepackage{mathrsfs}
    

    
    % Colors for the hyperref package
    \definecolor{urlcolor}{rgb}{0,.145,.698}
    \definecolor{linkcolor}{rgb}{.71,0.21,0.01}
    \definecolor{citecolor}{rgb}{.12,.54,.11}

    % ANSI colors
    \definecolor{ansi-black}{HTML}{3E424D}
    \definecolor{ansi-black-intense}{HTML}{282C36}
    \definecolor{ansi-red}{HTML}{E75C58}
    \definecolor{ansi-red-intense}{HTML}{B22B31}
    \definecolor{ansi-green}{HTML}{00A250}
    \definecolor{ansi-green-intense}{HTML}{007427}
    \definecolor{ansi-yellow}{HTML}{DDB62B}
    \definecolor{ansi-yellow-intense}{HTML}{B27D12}
    \definecolor{ansi-blue}{HTML}{208FFB}
    \definecolor{ansi-blue-intense}{HTML}{0065CA}
    \definecolor{ansi-magenta}{HTML}{D160C4}
    \definecolor{ansi-magenta-intense}{HTML}{A03196}
    \definecolor{ansi-cyan}{HTML}{60C6C8}
    \definecolor{ansi-cyan-intense}{HTML}{258F8F}
    \definecolor{ansi-white}{HTML}{C5C1B4}
    \definecolor{ansi-white-intense}{HTML}{A1A6B2}
    \definecolor{ansi-default-inverse-fg}{HTML}{FFFFFF}
    \definecolor{ansi-default-inverse-bg}{HTML}{000000}

    % common color for the border for error outputs.
    \definecolor{outerrorbackground}{HTML}{FFDFDF}

    % commands and environments needed by pandoc snippets
    % extracted from the output of `pandoc -s`
    \providecommand{\tightlist}{%
      \setlength{\itemsep}{0pt}\setlength{\parskip}{0pt}}
    \DefineVerbatimEnvironment{Highlighting}{Verbatim}{commandchars=\\\{\}}
    % Add ',fontsize=\small' for more characters per line
    \newenvironment{Shaded}{}{}
    \newcommand{\KeywordTok}[1]{\textcolor[rgb]{0.00,0.44,0.13}{\textbf{{#1}}}}
    \newcommand{\DataTypeTok}[1]{\textcolor[rgb]{0.56,0.13,0.00}{{#1}}}
    \newcommand{\DecValTok}[1]{\textcolor[rgb]{0.25,0.63,0.44}{{#1}}}
    \newcommand{\BaseNTok}[1]{\textcolor[rgb]{0.25,0.63,0.44}{{#1}}}
    \newcommand{\FloatTok}[1]{\textcolor[rgb]{0.25,0.63,0.44}{{#1}}}
    \newcommand{\CharTok}[1]{\textcolor[rgb]{0.25,0.44,0.63}{{#1}}}
    \newcommand{\StringTok}[1]{\textcolor[rgb]{0.25,0.44,0.63}{{#1}}}
    \newcommand{\CommentTok}[1]{\textcolor[rgb]{0.38,0.63,0.69}{\textit{{#1}}}}
    \newcommand{\OtherTok}[1]{\textcolor[rgb]{0.00,0.44,0.13}{{#1}}}
    \newcommand{\AlertTok}[1]{\textcolor[rgb]{1.00,0.00,0.00}{\textbf{{#1}}}}
    \newcommand{\FunctionTok}[1]{\textcolor[rgb]{0.02,0.16,0.49}{{#1}}}
    \newcommand{\RegionMarkerTok}[1]{{#1}}
    \newcommand{\ErrorTok}[1]{\textcolor[rgb]{1.00,0.00,0.00}{\textbf{{#1}}}}
    \newcommand{\NormalTok}[1]{{#1}}

    % Additional commands for more recent versions of Pandoc
    \newcommand{\ConstantTok}[1]{\textcolor[rgb]{0.53,0.00,0.00}{{#1}}}
    \newcommand{\SpecialCharTok}[1]{\textcolor[rgb]{0.25,0.44,0.63}{{#1}}}
    \newcommand{\VerbatimStringTok}[1]{\textcolor[rgb]{0.25,0.44,0.63}{{#1}}}
    \newcommand{\SpecialStringTok}[1]{\textcolor[rgb]{0.73,0.40,0.53}{{#1}}}
    \newcommand{\ImportTok}[1]{{#1}}
    \newcommand{\DocumentationTok}[1]{\textcolor[rgb]{0.73,0.13,0.13}{\textit{{#1}}}}
    \newcommand{\AnnotationTok}[1]{\textcolor[rgb]{0.38,0.63,0.69}{\textbf{\textit{{#1}}}}}
    \newcommand{\CommentVarTok}[1]{\textcolor[rgb]{0.38,0.63,0.69}{\textbf{\textit{{#1}}}}}
    \newcommand{\VariableTok}[1]{\textcolor[rgb]{0.10,0.09,0.49}{{#1}}}
    \newcommand{\ControlFlowTok}[1]{\textcolor[rgb]{0.00,0.44,0.13}{\textbf{{#1}}}}
    \newcommand{\OperatorTok}[1]{\textcolor[rgb]{0.40,0.40,0.40}{{#1}}}
    \newcommand{\BuiltInTok}[1]{{#1}}
    \newcommand{\ExtensionTok}[1]{{#1}}
    \newcommand{\PreprocessorTok}[1]{\textcolor[rgb]{0.74,0.48,0.00}{{#1}}}
    \newcommand{\AttributeTok}[1]{\textcolor[rgb]{0.49,0.56,0.16}{{#1}}}
    \newcommand{\InformationTok}[1]{\textcolor[rgb]{0.38,0.63,0.69}{\textbf{\textit{{#1}}}}}
    \newcommand{\WarningTok}[1]{\textcolor[rgb]{0.38,0.63,0.69}{\textbf{\textit{{#1}}}}}


    % Define a nice break command that doesn't care if a line doesn't already
    % exist.
    \def\br{\hspace*{\fill} \\* }
    % Math Jax compatibility definitions
    \def\gt{>}
    \def\lt{<}
    \let\Oldtex\TeX
    \let\Oldlatex\LaTeX
    \renewcommand{\TeX}{\textrm{\Oldtex}}
    \renewcommand{\LaTeX}{\textrm{\Oldlatex}}
    % Document parameters
    % Document title
    \title{main}
    
    
    
    
    
    
    
% Pygments definitions
\makeatletter
\def\PY@reset{\let\PY@it=\relax \let\PY@bf=\relax%
    \let\PY@ul=\relax \let\PY@tc=\relax%
    \let\PY@bc=\relax \let\PY@ff=\relax}
\def\PY@tok#1{\csname PY@tok@#1\endcsname}
\def\PY@toks#1+{\ifx\relax#1\empty\else%
    \PY@tok{#1}\expandafter\PY@toks\fi}
\def\PY@do#1{\PY@bc{\PY@tc{\PY@ul{%
    \PY@it{\PY@bf{\PY@ff{#1}}}}}}}
\def\PY#1#2{\PY@reset\PY@toks#1+\relax+\PY@do{#2}}

\@namedef{PY@tok@w}{\def\PY@tc##1{\textcolor[rgb]{0.73,0.73,0.73}{##1}}}
\@namedef{PY@tok@c}{\let\PY@it=\textit\def\PY@tc##1{\textcolor[rgb]{0.24,0.48,0.48}{##1}}}
\@namedef{PY@tok@cp}{\def\PY@tc##1{\textcolor[rgb]{0.61,0.40,0.00}{##1}}}
\@namedef{PY@tok@k}{\let\PY@bf=\textbf\def\PY@tc##1{\textcolor[rgb]{0.00,0.50,0.00}{##1}}}
\@namedef{PY@tok@kp}{\def\PY@tc##1{\textcolor[rgb]{0.00,0.50,0.00}{##1}}}
\@namedef{PY@tok@kt}{\def\PY@tc##1{\textcolor[rgb]{0.69,0.00,0.25}{##1}}}
\@namedef{PY@tok@o}{\def\PY@tc##1{\textcolor[rgb]{0.40,0.40,0.40}{##1}}}
\@namedef{PY@tok@ow}{\let\PY@bf=\textbf\def\PY@tc##1{\textcolor[rgb]{0.67,0.13,1.00}{##1}}}
\@namedef{PY@tok@nb}{\def\PY@tc##1{\textcolor[rgb]{0.00,0.50,0.00}{##1}}}
\@namedef{PY@tok@nf}{\def\PY@tc##1{\textcolor[rgb]{0.00,0.00,1.00}{##1}}}
\@namedef{PY@tok@nc}{\let\PY@bf=\textbf\def\PY@tc##1{\textcolor[rgb]{0.00,0.00,1.00}{##1}}}
\@namedef{PY@tok@nn}{\let\PY@bf=\textbf\def\PY@tc##1{\textcolor[rgb]{0.00,0.00,1.00}{##1}}}
\@namedef{PY@tok@ne}{\let\PY@bf=\textbf\def\PY@tc##1{\textcolor[rgb]{0.80,0.25,0.22}{##1}}}
\@namedef{PY@tok@nv}{\def\PY@tc##1{\textcolor[rgb]{0.10,0.09,0.49}{##1}}}
\@namedef{PY@tok@no}{\def\PY@tc##1{\textcolor[rgb]{0.53,0.00,0.00}{##1}}}
\@namedef{PY@tok@nl}{\def\PY@tc##1{\textcolor[rgb]{0.46,0.46,0.00}{##1}}}
\@namedef{PY@tok@ni}{\let\PY@bf=\textbf\def\PY@tc##1{\textcolor[rgb]{0.44,0.44,0.44}{##1}}}
\@namedef{PY@tok@na}{\def\PY@tc##1{\textcolor[rgb]{0.41,0.47,0.13}{##1}}}
\@namedef{PY@tok@nt}{\let\PY@bf=\textbf\def\PY@tc##1{\textcolor[rgb]{0.00,0.50,0.00}{##1}}}
\@namedef{PY@tok@nd}{\def\PY@tc##1{\textcolor[rgb]{0.67,0.13,1.00}{##1}}}
\@namedef{PY@tok@s}{\def\PY@tc##1{\textcolor[rgb]{0.73,0.13,0.13}{##1}}}
\@namedef{PY@tok@sd}{\let\PY@it=\textit\def\PY@tc##1{\textcolor[rgb]{0.73,0.13,0.13}{##1}}}
\@namedef{PY@tok@si}{\let\PY@bf=\textbf\def\PY@tc##1{\textcolor[rgb]{0.64,0.35,0.47}{##1}}}
\@namedef{PY@tok@se}{\let\PY@bf=\textbf\def\PY@tc##1{\textcolor[rgb]{0.67,0.36,0.12}{##1}}}
\@namedef{PY@tok@sr}{\def\PY@tc##1{\textcolor[rgb]{0.64,0.35,0.47}{##1}}}
\@namedef{PY@tok@ss}{\def\PY@tc##1{\textcolor[rgb]{0.10,0.09,0.49}{##1}}}
\@namedef{PY@tok@sx}{\def\PY@tc##1{\textcolor[rgb]{0.00,0.50,0.00}{##1}}}
\@namedef{PY@tok@m}{\def\PY@tc##1{\textcolor[rgb]{0.40,0.40,0.40}{##1}}}
\@namedef{PY@tok@gh}{\let\PY@bf=\textbf\def\PY@tc##1{\textcolor[rgb]{0.00,0.00,0.50}{##1}}}
\@namedef{PY@tok@gu}{\let\PY@bf=\textbf\def\PY@tc##1{\textcolor[rgb]{0.50,0.00,0.50}{##1}}}
\@namedef{PY@tok@gd}{\def\PY@tc##1{\textcolor[rgb]{0.63,0.00,0.00}{##1}}}
\@namedef{PY@tok@gi}{\def\PY@tc##1{\textcolor[rgb]{0.00,0.52,0.00}{##1}}}
\@namedef{PY@tok@gr}{\def\PY@tc##1{\textcolor[rgb]{0.89,0.00,0.00}{##1}}}
\@namedef{PY@tok@ge}{\let\PY@it=\textit}
\@namedef{PY@tok@gs}{\let\PY@bf=\textbf}
\@namedef{PY@tok@ges}{\let\PY@bf=\textbf\let\PY@it=\textit}
\@namedef{PY@tok@gp}{\let\PY@bf=\textbf\def\PY@tc##1{\textcolor[rgb]{0.00,0.00,0.50}{##1}}}
\@namedef{PY@tok@go}{\def\PY@tc##1{\textcolor[rgb]{0.44,0.44,0.44}{##1}}}
\@namedef{PY@tok@gt}{\def\PY@tc##1{\textcolor[rgb]{0.00,0.27,0.87}{##1}}}
\@namedef{PY@tok@err}{\def\PY@bc##1{{\setlength{\fboxsep}{\string -\fboxrule}\fcolorbox[rgb]{1.00,0.00,0.00}{1,1,1}{\strut ##1}}}}
\@namedef{PY@tok@kc}{\let\PY@bf=\textbf\def\PY@tc##1{\textcolor[rgb]{0.00,0.50,0.00}{##1}}}
\@namedef{PY@tok@kd}{\let\PY@bf=\textbf\def\PY@tc##1{\textcolor[rgb]{0.00,0.50,0.00}{##1}}}
\@namedef{PY@tok@kn}{\let\PY@bf=\textbf\def\PY@tc##1{\textcolor[rgb]{0.00,0.50,0.00}{##1}}}
\@namedef{PY@tok@kr}{\let\PY@bf=\textbf\def\PY@tc##1{\textcolor[rgb]{0.00,0.50,0.00}{##1}}}
\@namedef{PY@tok@bp}{\def\PY@tc##1{\textcolor[rgb]{0.00,0.50,0.00}{##1}}}
\@namedef{PY@tok@fm}{\def\PY@tc##1{\textcolor[rgb]{0.00,0.00,1.00}{##1}}}
\@namedef{PY@tok@vc}{\def\PY@tc##1{\textcolor[rgb]{0.10,0.09,0.49}{##1}}}
\@namedef{PY@tok@vg}{\def\PY@tc##1{\textcolor[rgb]{0.10,0.09,0.49}{##1}}}
\@namedef{PY@tok@vi}{\def\PY@tc##1{\textcolor[rgb]{0.10,0.09,0.49}{##1}}}
\@namedef{PY@tok@vm}{\def\PY@tc##1{\textcolor[rgb]{0.10,0.09,0.49}{##1}}}
\@namedef{PY@tok@sa}{\def\PY@tc##1{\textcolor[rgb]{0.73,0.13,0.13}{##1}}}
\@namedef{PY@tok@sb}{\def\PY@tc##1{\textcolor[rgb]{0.73,0.13,0.13}{##1}}}
\@namedef{PY@tok@sc}{\def\PY@tc##1{\textcolor[rgb]{0.73,0.13,0.13}{##1}}}
\@namedef{PY@tok@dl}{\def\PY@tc##1{\textcolor[rgb]{0.73,0.13,0.13}{##1}}}
\@namedef{PY@tok@s2}{\def\PY@tc##1{\textcolor[rgb]{0.73,0.13,0.13}{##1}}}
\@namedef{PY@tok@sh}{\def\PY@tc##1{\textcolor[rgb]{0.73,0.13,0.13}{##1}}}
\@namedef{PY@tok@s1}{\def\PY@tc##1{\textcolor[rgb]{0.73,0.13,0.13}{##1}}}
\@namedef{PY@tok@mb}{\def\PY@tc##1{\textcolor[rgb]{0.40,0.40,0.40}{##1}}}
\@namedef{PY@tok@mf}{\def\PY@tc##1{\textcolor[rgb]{0.40,0.40,0.40}{##1}}}
\@namedef{PY@tok@mh}{\def\PY@tc##1{\textcolor[rgb]{0.40,0.40,0.40}{##1}}}
\@namedef{PY@tok@mi}{\def\PY@tc##1{\textcolor[rgb]{0.40,0.40,0.40}{##1}}}
\@namedef{PY@tok@il}{\def\PY@tc##1{\textcolor[rgb]{0.40,0.40,0.40}{##1}}}
\@namedef{PY@tok@mo}{\def\PY@tc##1{\textcolor[rgb]{0.40,0.40,0.40}{##1}}}
\@namedef{PY@tok@ch}{\let\PY@it=\textit\def\PY@tc##1{\textcolor[rgb]{0.24,0.48,0.48}{##1}}}
\@namedef{PY@tok@cm}{\let\PY@it=\textit\def\PY@tc##1{\textcolor[rgb]{0.24,0.48,0.48}{##1}}}
\@namedef{PY@tok@cpf}{\let\PY@it=\textit\def\PY@tc##1{\textcolor[rgb]{0.24,0.48,0.48}{##1}}}
\@namedef{PY@tok@c1}{\let\PY@it=\textit\def\PY@tc##1{\textcolor[rgb]{0.24,0.48,0.48}{##1}}}
\@namedef{PY@tok@cs}{\let\PY@it=\textit\def\PY@tc##1{\textcolor[rgb]{0.24,0.48,0.48}{##1}}}

\def\PYZbs{\char`\\}
\def\PYZus{\char`\_}
\def\PYZob{\char`\{}
\def\PYZcb{\char`\}}
\def\PYZca{\char`\^}
\def\PYZam{\char`\&}
\def\PYZlt{\char`\<}
\def\PYZgt{\char`\>}
\def\PYZsh{\char`\#}
\def\PYZpc{\char`\%}
\def\PYZdl{\char`\$}
\def\PYZhy{\char`\-}
\def\PYZsq{\char`\'}
\def\PYZdq{\char`\"}
\def\PYZti{\char`\~}
% for compatibility with earlier versions
\def\PYZat{@}
\def\PYZlb{[}
\def\PYZrb{]}
\makeatother


    % For linebreaks inside Verbatim environment from package fancyvrb.
    \makeatletter
        \newbox\Wrappedcontinuationbox
        \newbox\Wrappedvisiblespacebox
        \newcommand*\Wrappedvisiblespace {\textcolor{red}{\textvisiblespace}}
        \newcommand*\Wrappedcontinuationsymbol {\textcolor{red}{\llap{\tiny$\m@th\hookrightarrow$}}}
        \newcommand*\Wrappedcontinuationindent {3ex }
        \newcommand*\Wrappedafterbreak {\kern\Wrappedcontinuationindent\copy\Wrappedcontinuationbox}
        % Take advantage of the already applied Pygments mark-up to insert
        % potential linebreaks for TeX processing.
        %        {, <, #, %, $, ' and ": go to next line.
        %        _, }, ^, &, >, - and ~: stay at end of broken line.
        % Use of \textquotesingle for straight quote.
        \newcommand*\Wrappedbreaksatspecials {%
            \def\PYGZus{\discretionary{\char`\_}{\Wrappedafterbreak}{\char`\_}}%
            \def\PYGZob{\discretionary{}{\Wrappedafterbreak\char`\{}{\char`\{}}%
            \def\PYGZcb{\discretionary{\char`\}}{\Wrappedafterbreak}{\char`\}}}%
            \def\PYGZca{\discretionary{\char`\^}{\Wrappedafterbreak}{\char`\^}}%
            \def\PYGZam{\discretionary{\char`\&}{\Wrappedafterbreak}{\char`\&}}%
            \def\PYGZlt{\discretionary{}{\Wrappedafterbreak\char`\<}{\char`\<}}%
            \def\PYGZgt{\discretionary{\char`\>}{\Wrappedafterbreak}{\char`\>}}%
            \def\PYGZsh{\discretionary{}{\Wrappedafterbreak\char`\#}{\char`\#}}%
            \def\PYGZpc{\discretionary{}{\Wrappedafterbreak\char`\%}{\char`\%}}%
            \def\PYGZdl{\discretionary{}{\Wrappedafterbreak\char`\$}{\char`\$}}%
            \def\PYGZhy{\discretionary{\char`\-}{\Wrappedafterbreak}{\char`\-}}%
            \def\PYGZsq{\discretionary{}{\Wrappedafterbreak\textquotesingle}{\textquotesingle}}%
            \def\PYGZdq{\discretionary{}{\Wrappedafterbreak\char`\"}{\char`\"}}%
            \def\PYGZti{\discretionary{\char`\~}{\Wrappedafterbreak}{\char`\~}}%
        }
        % Some characters . , ; ? ! / are not pygmentized.
        % This macro makes them "active" and they will insert potential linebreaks
        \newcommand*\Wrappedbreaksatpunct {%
            \lccode`\~`\.\lowercase{\def~}{\discretionary{\hbox{\char`\.}}{\Wrappedafterbreak}{\hbox{\char`\.}}}%
            \lccode`\~`\,\lowercase{\def~}{\discretionary{\hbox{\char`\,}}{\Wrappedafterbreak}{\hbox{\char`\,}}}%
            \lccode`\~`\;\lowercase{\def~}{\discretionary{\hbox{\char`\;}}{\Wrappedafterbreak}{\hbox{\char`\;}}}%
            \lccode`\~`\:\lowercase{\def~}{\discretionary{\hbox{\char`\:}}{\Wrappedafterbreak}{\hbox{\char`\:}}}%
            \lccode`\~`\?\lowercase{\def~}{\discretionary{\hbox{\char`\?}}{\Wrappedafterbreak}{\hbox{\char`\?}}}%
            \lccode`\~`\!\lowercase{\def~}{\discretionary{\hbox{\char`\!}}{\Wrappedafterbreak}{\hbox{\char`\!}}}%
            \lccode`\~`\/\lowercase{\def~}{\discretionary{\hbox{\char`\/}}{\Wrappedafterbreak}{\hbox{\char`\/}}}%
            \catcode`\.\active
            \catcode`\,\active
            \catcode`\;\active
            \catcode`\:\active
            \catcode`\?\active
            \catcode`\!\active
            \catcode`\/\active
            \lccode`\~`\~
        }
    \makeatother

    \let\OriginalVerbatim=\Verbatim
    \makeatletter
    \renewcommand{\Verbatim}[1][1]{%
        %\parskip\z@skip
        \sbox\Wrappedcontinuationbox {\Wrappedcontinuationsymbol}%
        \sbox\Wrappedvisiblespacebox {\FV@SetupFont\Wrappedvisiblespace}%
        \def\FancyVerbFormatLine ##1{\hsize\linewidth
            \vtop{\raggedright\hyphenpenalty\z@\exhyphenpenalty\z@
                \doublehyphendemerits\z@\finalhyphendemerits\z@
                \strut ##1\strut}%
        }%
        % If the linebreak is at a space, the latter will be displayed as visible
        % space at end of first line, and a continuation symbol starts next line.
        % Stretch/shrink are however usually zero for typewriter font.
        \def\FV@Space {%
            \nobreak\hskip\z@ plus\fontdimen3\font minus\fontdimen4\font
            \discretionary{\copy\Wrappedvisiblespacebox}{\Wrappedafterbreak}
            {\kern\fontdimen2\font}%
        }%

        % Allow breaks at special characters using \PYG... macros.
        \Wrappedbreaksatspecials
        % Breaks at punctuation characters . , ; ? ! and / need catcode=\active
        \OriginalVerbatim[#1,codes*=\Wrappedbreaksatpunct]%
    }
    \makeatother

    % Exact colors from NB
    \definecolor{incolor}{HTML}{303F9F}
    \definecolor{outcolor}{HTML}{D84315}
    \definecolor{cellborder}{HTML}{CFCFCF}
    \definecolor{cellbackground}{HTML}{F7F7F7}

    % prompt
    \makeatletter
    \newcommand{\boxspacing}{\kern\kvtcb@left@rule\kern\kvtcb@boxsep}
    \makeatother
    \newcommand{\prompt}[4]{
        {\ttfamily\llap{{\color{#2}[#3]:\hspace{3pt}#4}}\vspace{-\baselineskip}}
    }
    

    
    % Prevent overflowing lines due to hard-to-break entities
    \sloppy
    % Setup hyperref package
    \hypersetup{
      breaklinks=true,  % so long urls are correctly broken across lines
      colorlinks=true,
      urlcolor=urlcolor,
      linkcolor=linkcolor,
      citecolor=citecolor,
      }
    % Slightly bigger margins than the latex defaults
    
    \geometry{verbose,tmargin=1in,bmargin=1in,lmargin=1in,rmargin=1in}
    
    

\begin{document}
    
    \maketitle
    
    

    
    \begin{tcolorbox}[breakable, size=fbox, boxrule=1pt, pad at break*=1mm,colback=cellbackground, colframe=cellborder]
\prompt{In}{incolor}{76}{\boxspacing}
\begin{Verbatim}[commandchars=\\\{\}]
\PY{k+kn}{import} \PY{n+nn}{subprocess}

\PY{c+c1}{\PYZsh{} Startet `ollama serve` im Hintergrund}
\PY{n}{process} \PY{o}{=} \PY{n}{subprocess}\PY{o}{.}\PY{n}{Popen}\PY{p}{(}\PY{p}{[}\PY{l+s+s2}{\PYZdq{}}\PY{l+s+s2}{ollama}\PY{l+s+s2}{\PYZdq{}}\PY{p}{,} \PY{l+s+s2}{\PYZdq{}}\PY{l+s+s2}{serve}\PY{l+s+s2}{\PYZdq{}}\PY{p}{]}\PY{p}{,} \PY{n}{stdout}\PY{o}{=}\PY{n}{subprocess}\PY{o}{.}\PY{n}{PIPE}\PY{p}{,} \PY{n}{stderr}\PY{o}{=}\PY{n}{subprocess}\PY{o}{.}\PY{n}{PIPE}\PY{p}{)}
\PY{n+nb}{print}\PY{p}{(}\PY{l+s+s2}{\PYZdq{}}\PY{l+s+s2}{Ollama server is now running in the background.}\PY{l+s+s2}{\PYZdq{}}\PY{p}{)} 
\end{Verbatim}
\end{tcolorbox}

    \begin{Verbatim}[commandchars=\\\{\}]
Ollama server is now running in the background.
    \end{Verbatim}

    \begin{tcolorbox}[breakable, size=fbox, boxrule=1pt, pad at break*=1mm,colback=cellbackground, colframe=cellborder]
\prompt{In}{incolor}{77}{\boxspacing}
\begin{Verbatim}[commandchars=\\\{\}]
\PY{c+c1}{\PYZsh{} Lege das LLM fest, mit dem nachfolgend gearbeitet wird.}
\PY{n}{model} \PY{o}{=} \PY{l+s+s2}{\PYZdq{}}\PY{l+s+s2}{qwen2.5:7b}\PY{l+s+s2}{\PYZdq{}}
\PY{n}{model\PYZus{}custom} \PY{o}{=} \PY{l+s+s2}{\PYZdq{}}\PY{l+s+s2}{custom\PYZus{}qwen2.5:7b\PYZus{}seed\PYZus{}42\PYZus{}temp\PYZus{}0}\PY{l+s+s2}{\PYZdq{}}
\end{Verbatim}
\end{tcolorbox}

    \begin{tcolorbox}[breakable, size=fbox, boxrule=1pt, pad at break*=1mm,colback=cellbackground, colframe=cellborder]
\prompt{In}{incolor}{78}{\boxspacing}
\begin{Verbatim}[commandchars=\\\{\}]
\PY{c+c1}{\PYZsh{} Code zum Erstellen eines Modells mit festgelegtem Seed und Temperatur in der Modelfile}

\PY{c+c1}{\PYZsh{} Definieren des Inhalts der Modelfile mit einem spezifischen Modell, Seed und Temperatur}
\PY{n}{seed\PYZus{}value} \PY{o}{=} \PY{l+m+mi}{42}  \PY{c+c1}{\PYZsh{} Beispiel\PYZhy{}Seed\PYZhy{}Wert für reproduzierbare Ergebnisse}
\PY{n}{temperature\PYZus{}value} \PY{o}{=} \PY{l+m+mi}{0}  \PY{c+c1}{\PYZsh{} Temperatur auf 0 setzen für deterministische Ausgaben}

\PY{c+c1}{\PYZsh{} Inhalt der Modelfile mit Seed\PYZhy{} und Temperaturparametern}
\PY{n}{modelfile\PYZus{}content} \PY{o}{=} \PY{l+s+sa}{f}\PY{l+s+s2}{\PYZdq{}}\PY{l+s+s2}{FROM }\PY{l+s+si}{\PYZob{}}\PY{n}{model}\PY{l+s+si}{\PYZcb{}}\PY{l+s+se}{\PYZbs{}n}\PY{l+s+s2}{PARAMETER seed }\PY{l+s+si}{\PYZob{}}\PY{n}{seed\PYZus{}value}\PY{l+s+si}{\PYZcb{}}\PY{l+s+se}{\PYZbs{}n}\PY{l+s+s2}{PARAMETER temperature }\PY{l+s+si}{\PYZob{}}\PY{n}{temperature\PYZus{}value}\PY{l+s+si}{\PYZcb{}}\PY{l+s+se}{\PYZbs{}n}\PY{l+s+s2}{\PYZdq{}}

\PY{c+c1}{\PYZsh{} Schreiben des Inhalts in eine Modelfile}
\PY{n}{modelfile\PYZus{}path} \PY{o}{=} \PY{l+s+s2}{\PYZdq{}}\PY{l+s+s2}{Modelfile}\PY{l+s+s2}{\PYZdq{}}  \PY{c+c1}{\PYZsh{} Pfad, um die Modelfile im aktuellen Verzeichnis zu speichern}

\PY{k}{with} \PY{n+nb}{open}\PY{p}{(}\PY{n}{modelfile\PYZus{}path}\PY{p}{,} \PY{l+s+s2}{\PYZdq{}}\PY{l+s+s2}{w}\PY{l+s+s2}{\PYZdq{}}\PY{p}{)} \PY{k}{as} \PY{n}{modelfile}\PY{p}{:}
    \PY{n}{modelfile}\PY{o}{.}\PY{n}{write}\PY{p}{(}\PY{n}{modelfile\PYZus{}content}\PY{p}{)}

\PY{c+c1}{\PYZsh{} Befehl zum Erstellen des Modells mithilfe der angepassten Modelfile}
\PY{n}{command} \PY{o}{=} \PY{p}{[}\PY{l+s+s2}{\PYZdq{}}\PY{l+s+s2}{ollama}\PY{l+s+s2}{\PYZdq{}}\PY{p}{,} \PY{l+s+s2}{\PYZdq{}}\PY{l+s+s2}{create}\PY{l+s+s2}{\PYZdq{}}\PY{p}{,} \PY{l+s+s2}{\PYZdq{}}\PY{l+s+s2}{custom\PYZus{}qwen2.5:7b\PYZus{}seed\PYZus{}42\PYZus{}temp\PYZus{}0}\PY{l+s+s2}{\PYZdq{}}\PY{p}{,} \PY{l+s+s2}{\PYZdq{}}\PY{l+s+s2}{\PYZhy{}f}\PY{l+s+s2}{\PYZdq{}}\PY{p}{,} \PY{n}{modelfile\PYZus{}path}\PY{p}{]}

\PY{c+c1}{\PYZsh{} Führen Sie den Befehl aus, um das Modell zu erstellen}
\PY{k}{try}\PY{p}{:}
    \PY{n}{result} \PY{o}{=} \PY{n}{subprocess}\PY{o}{.}\PY{n}{run}\PY{p}{(}\PY{n}{command}\PY{p}{,} \PY{n}{capture\PYZus{}output}\PY{o}{=}\PY{k+kc}{True}\PY{p}{,} \PY{n}{text}\PY{o}{=}\PY{k+kc}{True}\PY{p}{,} \PY{n}{check}\PY{o}{=}\PY{k+kc}{True}\PY{p}{)}
    \PY{n+nb}{print}\PY{p}{(}\PY{n}{result}\PY{o}{.}\PY{n}{stdout}\PY{p}{)}  \PY{c+c1}{\PYZsh{} Gibt die Standardausgabe aus, falls erfolgreich}
\PY{k}{except} \PY{n}{subprocess}\PY{o}{.}\PY{n}{CalledProcessError} \PY{k}{as} \PY{n}{e}\PY{p}{:}
    \PY{n+nb}{print}\PY{p}{(}\PY{l+s+s2}{\PYZdq{}}\PY{l+s+s2}{An error occurred:}\PY{l+s+s2}{\PYZdq{}}\PY{p}{,} \PY{n}{e}\PY{o}{.}\PY{n}{stderr}\PY{p}{)}  \PY{c+c1}{\PYZsh{} Zeigt eventuelle Fehlermeldungen an, falls der Befehl fehlschlägt}
\PY{k}{except} \PY{n+ne}{FileNotFoundError}\PY{p}{:}
    \PY{n+nb}{print}\PY{p}{(}\PY{l+s+s2}{\PYZdq{}}\PY{l+s+s2}{ERROR: }\PY{l+s+s2}{\PYZsq{}}\PY{l+s+s2}{ollama}\PY{l+s+s2}{\PYZsq{}}\PY{l+s+s2}{ command not found. Ensure that the Ollama CLI is installed and accessible.}\PY{l+s+s2}{\PYZdq{}}\PY{p}{)}
\end{Verbatim}
\end{tcolorbox}

    \begin{Verbatim}[commandchars=\\\{\}]

    \end{Verbatim}

    \begin{tcolorbox}[breakable, size=fbox, boxrule=1pt, pad at break*=1mm,colback=cellbackground, colframe=cellborder]
\prompt{In}{incolor}{79}{\boxspacing}
\begin{Verbatim}[commandchars=\\\{\}]
\PY{c+c1}{\PYZsh{} Gewähltes Modell von Ollama herunterladen}
\PY{n}{process} \PY{o}{=} \PY{n}{subprocess}\PY{o}{.}\PY{n}{Popen}\PY{p}{(}\PY{p}{[}\PY{l+s+s2}{\PYZdq{}}\PY{l+s+s2}{ollama}\PY{l+s+s2}{\PYZdq{}}\PY{p}{,} \PY{l+s+s2}{\PYZdq{}}\PY{l+s+s2}{pull}\PY{l+s+s2}{\PYZdq{}}\PY{p}{,} \PY{n}{model}\PY{p}{]}\PY{p}{,} \PY{n}{stdout}\PY{o}{=}\PY{n}{subprocess}\PY{o}{.}\PY{n}{PIPE}\PY{p}{,} \PY{n}{stderr}\PY{o}{=}\PY{n}{subprocess}\PY{o}{.}\PY{n}{PIPE}\PY{p}{)}

\PY{c+c1}{\PYZsh{} Ausgabe anzeigen}
\PY{n}{stdout}\PY{p}{,} \PY{n}{stderr} \PY{o}{=} \PY{n}{process}\PY{o}{.}\PY{n}{communicate}\PY{p}{(}\PY{p}{)}
\PY{n+nb}{print}\PY{p}{(}\PY{n}{stdout}\PY{o}{.}\PY{n}{decode}\PY{p}{(}\PY{l+s+s2}{\PYZdq{}}\PY{l+s+s2}{utf\PYZhy{}8}\PY{l+s+s2}{\PYZdq{}}\PY{p}{)}\PY{p}{)}

\PY{k}{if} \PY{n}{stderr}\PY{p}{:}
    \PY{n+nb}{print}\PY{p}{(}\PY{l+s+s2}{\PYZdq{}}\PY{l+s+s2}{Error:}\PY{l+s+s2}{\PYZdq{}}\PY{p}{,} \PY{n}{stderr}\PY{o}{.}\PY{n}{decode}\PY{p}{(}\PY{l+s+s2}{\PYZdq{}}\PY{l+s+s2}{utf\PYZhy{}8}\PY{l+s+s2}{\PYZdq{}}\PY{p}{)}\PY{p}{)}
\PY{k}{else}\PY{p}{:}
    \PY{n+nb}{print}\PY{p}{(}\PY{l+s+sa}{f}\PY{l+s+s2}{\PYZdq{}}\PY{l+s+si}{\PYZob{}}\PY{n}{model}\PY{l+s+si}{\PYZcb{}}\PY{l+s+s2}{ pulled successfully.}\PY{l+s+s2}{\PYZdq{}}\PY{p}{)}
\end{Verbatim}
\end{tcolorbox}

    \begin{Verbatim}[commandchars=\\\{\}]

Error: pulling manifest ⠋ pulling manifest ⠹
pulling manifest ⠹ pulling manifest ⠸
pulling manifest ⠼ pulling manifest ⠴
pulling manifest
pulling 2bada8a74506{\ldots} 100\% ▕████████████████▏ 4.7 GB
pulling 66b9ea09bd5b{\ldots} 100\% ▕████████████████▏   68 B
pulling eb4402837c78{\ldots} 100\% ▕████████████████▏ 1.5 KB
pulling 832dd9e00a68{\ldots} 100\% ▕████████████████▏  11 KB
pulling 2f15b3218f05{\ldots} 100\% ▕████████████████▏  487 B
verifying sha256 digest
writing manifest
success 

    \end{Verbatim}

    \begin{tcolorbox}[breakable, size=fbox, boxrule=1pt, pad at break*=1mm,colback=cellbackground, colframe=cellborder]
\prompt{In}{incolor}{80}{\boxspacing}
\begin{Verbatim}[commandchars=\\\{\}]
\PY{k+kn}{import} \PY{n+nn}{faker}
\PY{k+kn}{import} \PY{n+nn}{pandas} \PY{k}{as} \PY{n+nn}{pd}
\PY{k+kn}{import} \PY{n+nn}{random}
\PY{k+kn}{from} \PY{n+nn}{faker} \PY{k+kn}{import} \PY{n}{Faker}
\PY{k+kn}{from} \PY{n+nn}{datetime} \PY{k+kn}{import} \PY{n}{datetime}\PY{p}{,} \PY{n}{timedelta}
\PY{k+kn}{import} \PY{n+nn}{os}

\PY{c+c1}{\PYZsh{} Seed setzen für Reproduzierbarkeit}
\PY{n}{random}\PY{o}{.}\PY{n}{seed}\PY{p}{(}\PY{l+m+mi}{42}\PY{p}{)}
\PY{n}{Faker}\PY{o}{.}\PY{n}{seed}\PY{p}{(}\PY{l+m+mi}{42}\PY{p}{)}
\PY{n}{fake} \PY{o}{=} \PY{n}{Faker}\PY{p}{(}\PY{p}{)}

\PY{c+c1}{\PYZsh{} Ordner \PYZsq{}tables\PYZsq{} erstellen, falls nicht vorhanden}
\PY{n}{output\PYZus{}folder} \PY{o}{=} \PY{l+s+s2}{\PYZdq{}}\PY{l+s+s2}{tables}\PY{l+s+s2}{\PYZdq{}}
\PY{n}{os}\PY{o}{.}\PY{n}{makedirs}\PY{p}{(}\PY{n}{output\PYZus{}folder}\PY{p}{,} \PY{n}{exist\PYZus{}ok}\PY{o}{=}\PY{k+kc}{True}\PY{p}{)}

\PY{c+c1}{\PYZsh{} Funktion, um zufällige Datumswerte zu erzeugen}
\PY{k}{def} \PY{n+nf}{random\PYZus{}date}\PY{p}{(}\PY{n}{start}\PY{p}{,} \PY{n}{end}\PY{p}{)}\PY{p}{:}
    \PY{k}{return} \PY{n}{start} \PY{o}{+} \PY{n}{timedelta}\PY{p}{(}\PY{n}{days}\PY{o}{=}\PY{n}{random}\PY{o}{.}\PY{n}{randint}\PY{p}{(}\PY{l+m+mi}{0}\PY{p}{,} \PY{p}{(}\PY{n}{end} \PY{o}{\PYZhy{}} \PY{n}{start}\PY{p}{)}\PY{o}{.}\PY{n}{days}\PY{p}{)}\PY{p}{)}

\PY{c+c1}{\PYZsh{} Anzahl der Datensätze, die für die Datenbank generiert werden sollen}
\PY{n}{num\PYZus{}students} \PY{o}{=} \PY{l+m+mi}{100}
\PY{n}{num\PYZus{}courses} \PY{o}{=} \PY{l+m+mi}{10}
\PY{n}{num\PYZus{}professors} \PY{o}{=} \PY{l+m+mi}{5}
\PY{n}{num\PYZus{}departments} \PY{o}{=} \PY{l+m+mi}{3}
\PY{n}{num\PYZus{}enrollments} \PY{o}{=} \PY{l+m+mi}{300}

\PY{c+c1}{\PYZsh{} 1. STUDENT\PYZus{}DIMENSION \PYZhy{} Generierung der Studentendaten}
\PY{n}{student\PYZus{}data} \PY{o}{=} \PY{p}{\PYZob{}}
    \PY{l+s+s2}{\PYZdq{}}\PY{l+s+s2}{Student\PYZus{}ID}\PY{l+s+s2}{\PYZdq{}}\PY{p}{:} \PY{n+nb}{range}\PY{p}{(}\PY{l+m+mi}{1}\PY{p}{,} \PY{n}{num\PYZus{}students} \PY{o}{+} \PY{l+m+mi}{1}\PY{p}{)}\PY{p}{,}
    \PY{l+s+s2}{\PYZdq{}}\PY{l+s+s2}{First\PYZus{}Name}\PY{l+s+s2}{\PYZdq{}}\PY{p}{:} \PY{p}{[}\PY{n}{fake}\PY{o}{.}\PY{n}{first\PYZus{}name}\PY{p}{(}\PY{p}{)} \PY{k}{for} \PY{n}{\PYZus{}} \PY{o+ow}{in} \PY{n+nb}{range}\PY{p}{(}\PY{n}{num\PYZus{}students}\PY{p}{)}\PY{p}{]}\PY{p}{,}
    \PY{l+s+s2}{\PYZdq{}}\PY{l+s+s2}{Last\PYZus{}Name}\PY{l+s+s2}{\PYZdq{}}\PY{p}{:} \PY{p}{[}\PY{n}{fake}\PY{o}{.}\PY{n}{last\PYZus{}name}\PY{p}{(}\PY{p}{)} \PY{k}{for} \PY{n}{\PYZus{}} \PY{o+ow}{in} \PY{n+nb}{range}\PY{p}{(}\PY{n}{num\PYZus{}students}\PY{p}{)}\PY{p}{]}\PY{p}{,}
    \PY{l+s+s2}{\PYZdq{}}\PY{l+s+s2}{Date\PYZus{}of\PYZus{}Birth}\PY{l+s+s2}{\PYZdq{}}\PY{p}{:} \PY{p}{[}\PY{n}{random\PYZus{}date}\PY{p}{(}\PY{n}{datetime}\PY{p}{(}\PY{l+m+mi}{1990}\PY{p}{,} \PY{l+m+mi}{1}\PY{p}{,} \PY{l+m+mi}{1}\PY{p}{)}\PY{p}{,} \PY{n}{datetime}\PY{p}{(}\PY{l+m+mi}{2002}\PY{p}{,} \PY{l+m+mi}{12}\PY{p}{,} \PY{l+m+mi}{31}\PY{p}{)}\PY{p}{)}\PY{o}{.}\PY{n}{date}\PY{p}{(}\PY{p}{)} \PY{k}{for} \PY{n}{\PYZus{}} \PY{o+ow}{in} \PY{n+nb}{range}\PY{p}{(}\PY{n}{num\PYZus{}students}\PY{p}{)}\PY{p}{]}\PY{p}{,}
    \PY{l+s+s2}{\PYZdq{}}\PY{l+s+s2}{Enrollment\PYZus{}Date}\PY{l+s+s2}{\PYZdq{}}\PY{p}{:} \PY{p}{[}\PY{n}{random\PYZus{}date}\PY{p}{(}\PY{n}{datetime}\PY{p}{(}\PY{l+m+mi}{2020}\PY{p}{,} \PY{l+m+mi}{1}\PY{p}{,} \PY{l+m+mi}{1}\PY{p}{)}\PY{p}{,} \PY{n}{datetime}\PY{p}{(}\PY{l+m+mi}{2023}\PY{p}{,} \PY{l+m+mi}{12}\PY{p}{,} \PY{l+m+mi}{31}\PY{p}{)}\PY{p}{)}\PY{o}{.}\PY{n}{date}\PY{p}{(}\PY{p}{)} \PY{k}{for} \PY{n}{\PYZus{}} \PY{o+ow}{in} \PY{n+nb}{range}\PY{p}{(}\PY{n}{num\PYZus{}students}\PY{p}{)}\PY{p}{]}
\PY{p}{\PYZcb{}}

\PY{c+c1}{\PYZsh{} Erstellung des DataFrames}
\PY{n}{student\PYZus{}df} \PY{o}{=} \PY{n}{pd}\PY{o}{.}\PY{n}{DataFrame}\PY{p}{(}\PY{n}{student\PYZus{}data}\PY{p}{)}

\PY{c+c1}{\PYZsh{} E\PYZhy{}Mail\PYZhy{}Adresse basierend auf First\PYZus{}Name und Last\PYZus{}Name generieren; wenn hier nochmal random verwendet wird, erhalten wir sonst abweichende Namen in der Mailadresse.}
\PY{n}{student\PYZus{}df}\PY{p}{[}\PY{l+s+s2}{\PYZdq{}}\PY{l+s+s2}{Email}\PY{l+s+s2}{\PYZdq{}}\PY{p}{]} \PY{o}{=} \PY{n}{student\PYZus{}df}\PY{p}{[}\PY{l+s+s2}{\PYZdq{}}\PY{l+s+s2}{First\PYZus{}Name}\PY{l+s+s2}{\PYZdq{}}\PY{p}{]}\PY{o}{.}\PY{n}{str}\PY{o}{.}\PY{n}{lower}\PY{p}{(}\PY{p}{)} \PY{o}{+} \PY{l+s+s2}{\PYZdq{}}\PY{l+s+s2}{.}\PY{l+s+s2}{\PYZdq{}} \PY{o}{+} \PY{n}{student\PYZus{}df}\PY{p}{[}\PY{l+s+s2}{\PYZdq{}}\PY{l+s+s2}{Last\PYZus{}Name}\PY{l+s+s2}{\PYZdq{}}\PY{p}{]}\PY{o}{.}\PY{n}{str}\PY{o}{.}\PY{n}{lower}\PY{p}{(}\PY{p}{)} \PY{o}{+} \PY{l+s+s2}{\PYZdq{}}\PY{l+s+s2}{@example.com}\PY{l+s+s2}{\PYZdq{}}

\PY{c+c1}{\PYZsh{} 2. COURSE\PYZus{}DIMENSION \PYZhy{} Generierung der Kursdaten}
\PY{n}{course\PYZus{}names} \PY{o}{=} \PY{p}{[}
    \PY{l+s+s2}{\PYZdq{}}\PY{l+s+s2}{Data Science Basics}\PY{l+s+s2}{\PYZdq{}}\PY{p}{,} \PY{l+s+s2}{\PYZdq{}}\PY{l+s+s2}{Advanced Machine Learning}\PY{l+s+s2}{\PYZdq{}}\PY{p}{,} \PY{l+s+s2}{\PYZdq{}}\PY{l+s+s2}{Database Systems}\PY{l+s+s2}{\PYZdq{}}\PY{p}{,}
    \PY{l+s+s2}{\PYZdq{}}\PY{l+s+s2}{Statistics for Data Science}\PY{l+s+s2}{\PYZdq{}}\PY{p}{,} \PY{l+s+s2}{\PYZdq{}}\PY{l+s+s2}{Programming with Python}\PY{l+s+s2}{\PYZdq{}}\PY{p}{,} \PY{l+s+s2}{\PYZdq{}}\PY{l+s+s2}{Ethics in AI}\PY{l+s+s2}{\PYZdq{}}\PY{p}{,}
    \PY{l+s+s2}{\PYZdq{}}\PY{l+s+s2}{Big Data Analysis}\PY{l+s+s2}{\PYZdq{}}\PY{p}{,} \PY{l+s+s2}{\PYZdq{}}\PY{l+s+s2}{Data Visualization}\PY{l+s+s2}{\PYZdq{}}\PY{p}{,} \PY{l+s+s2}{\PYZdq{}}\PY{l+s+s2}{Project Management}\PY{l+s+s2}{\PYZdq{}}\PY{p}{,} \PY{l+s+s2}{\PYZdq{}}\PY{l+s+s2}{Deep Learning}\PY{l+s+s2}{\PYZdq{}}
\PY{p}{]}
\PY{n}{course\PYZus{}data} \PY{o}{=} \PY{p}{\PYZob{}}
    \PY{l+s+s2}{\PYZdq{}}\PY{l+s+s2}{Course\PYZus{}ID}\PY{l+s+s2}{\PYZdq{}}\PY{p}{:} \PY{n+nb}{range}\PY{p}{(}\PY{l+m+mi}{1}\PY{p}{,} \PY{n}{num\PYZus{}courses} \PY{o}{+} \PY{l+m+mi}{1}\PY{p}{)}\PY{p}{,}
    \PY{l+s+s2}{\PYZdq{}}\PY{l+s+s2}{Course\PYZus{}Name}\PY{l+s+s2}{\PYZdq{}}\PY{p}{:} \PY{n}{course\PYZus{}names}\PY{p}{,}
    \PY{l+s+s2}{\PYZdq{}}\PY{l+s+s2}{Credits}\PY{l+s+s2}{\PYZdq{}}\PY{p}{:} \PY{p}{[}\PY{n}{random}\PY{o}{.}\PY{n}{choice}\PY{p}{(}\PY{p}{[}\PY{l+m+mi}{3}\PY{p}{,} \PY{l+m+mi}{4}\PY{p}{,} \PY{l+m+mi}{5}\PY{p}{]}\PY{p}{)} \PY{k}{for} \PY{n}{\PYZus{}} \PY{o+ow}{in} \PY{n+nb}{range}\PY{p}{(}\PY{n}{num\PYZus{}courses}\PY{p}{)}\PY{p}{]}\PY{p}{,}
    \PY{l+s+s2}{\PYZdq{}}\PY{l+s+s2}{Department\PYZus{}ID}\PY{l+s+s2}{\PYZdq{}}\PY{p}{:} \PY{p}{[}\PY{n}{random}\PY{o}{.}\PY{n}{randint}\PY{p}{(}\PY{l+m+mi}{1}\PY{p}{,} \PY{n}{num\PYZus{}departments}\PY{p}{)} \PY{k}{for} \PY{n}{\PYZus{}} \PY{o+ow}{in} \PY{n+nb}{range}\PY{p}{(}\PY{n}{num\PYZus{}courses}\PY{p}{)}\PY{p}{]}
\PY{p}{\PYZcb{}}
\PY{n}{course\PYZus{}df} \PY{o}{=} \PY{n}{pd}\PY{o}{.}\PY{n}{DataFrame}\PY{p}{(}\PY{n}{course\PYZus{}data}\PY{p}{)}

\PY{c+c1}{\PYZsh{} 3. PROFESSOR\PYZus{}DIMENSION \PYZhy{} Generierung der Professorendaten}
\PY{n}{professor\PYZus{}data} \PY{o}{=} \PY{p}{\PYZob{}}
    \PY{l+s+s2}{\PYZdq{}}\PY{l+s+s2}{Professor\PYZus{}ID}\PY{l+s+s2}{\PYZdq{}}\PY{p}{:} \PY{n+nb}{range}\PY{p}{(}\PY{l+m+mi}{1}\PY{p}{,} \PY{n}{num\PYZus{}professors} \PY{o}{+} \PY{l+m+mi}{1}\PY{p}{)}\PY{p}{,}
    \PY{l+s+s2}{\PYZdq{}}\PY{l+s+s2}{First\PYZus{}Name}\PY{l+s+s2}{\PYZdq{}}\PY{p}{:} \PY{p}{[}\PY{n}{fake}\PY{o}{.}\PY{n}{first\PYZus{}name}\PY{p}{(}\PY{p}{)} \PY{k}{for} \PY{n}{\PYZus{}} \PY{o+ow}{in} \PY{n+nb}{range}\PY{p}{(}\PY{n}{num\PYZus{}professors}\PY{p}{)}\PY{p}{]}\PY{p}{,}
    \PY{l+s+s2}{\PYZdq{}}\PY{l+s+s2}{Last\PYZus{}Name}\PY{l+s+s2}{\PYZdq{}}\PY{p}{:} \PY{p}{[}\PY{n}{fake}\PY{o}{.}\PY{n}{last\PYZus{}name}\PY{p}{(}\PY{p}{)} \PY{k}{for} \PY{n}{\PYZus{}} \PY{o+ow}{in} \PY{n+nb}{range}\PY{p}{(}\PY{n}{num\PYZus{}professors}\PY{p}{)}\PY{p}{]}\PY{p}{,}
    \PY{l+s+s2}{\PYZdq{}}\PY{l+s+s2}{Email}\PY{l+s+s2}{\PYZdq{}}\PY{p}{:} \PY{p}{[}\PY{l+s+sa}{f}\PY{l+s+s2}{\PYZdq{}}\PY{l+s+si}{\PYZob{}}\PY{n}{fake}\PY{o}{.}\PY{n}{first\PYZus{}name}\PY{p}{(}\PY{p}{)}\PY{o}{.}\PY{n}{lower}\PY{p}{(}\PY{p}{)}\PY{l+s+si}{\PYZcb{}}\PY{l+s+s2}{.}\PY{l+s+si}{\PYZob{}}\PY{n}{fake}\PY{o}{.}\PY{n}{last\PYZus{}name}\PY{p}{(}\PY{p}{)}\PY{o}{.}\PY{n}{lower}\PY{p}{(}\PY{p}{)}\PY{l+s+si}{\PYZcb{}}\PY{l+s+s2}{@university.com}\PY{l+s+s2}{\PYZdq{}} \PY{k}{for} \PY{n}{\PYZus{}} \PY{o+ow}{in} \PY{n+nb}{range}\PY{p}{(}\PY{n}{num\PYZus{}professors}\PY{p}{)}\PY{p}{]}
\PY{p}{\PYZcb{}}
\PY{n}{professor\PYZus{}df} \PY{o}{=} \PY{n}{pd}\PY{o}{.}\PY{n}{DataFrame}\PY{p}{(}\PY{n}{professor\PYZus{}data}\PY{p}{)}

\PY{c+c1}{\PYZsh{} 4. DEPARTMENT\PYZus{}DIMENSION \PYZhy{} Generierung der Abteilungsdaten}
\PY{n}{department\PYZus{}names} \PY{o}{=} \PY{p}{[}\PY{l+s+s2}{\PYZdq{}}\PY{l+s+s2}{Computer Science}\PY{l+s+s2}{\PYZdq{}}\PY{p}{,} \PY{l+s+s2}{\PYZdq{}}\PY{l+s+s2}{Business Administration}\PY{l+s+s2}{\PYZdq{}}\PY{p}{,} \PY{l+s+s2}{\PYZdq{}}\PY{l+s+s2}{Psychology}\PY{l+s+s2}{\PYZdq{}}\PY{p}{]}
\PY{n}{department\PYZus{}data} \PY{o}{=} \PY{p}{\PYZob{}}
    \PY{l+s+s2}{\PYZdq{}}\PY{l+s+s2}{Department\PYZus{}ID}\PY{l+s+s2}{\PYZdq{}}\PY{p}{:} \PY{n+nb}{range}\PY{p}{(}\PY{l+m+mi}{1}\PY{p}{,} \PY{n}{num\PYZus{}departments} \PY{o}{+} \PY{l+m+mi}{1}\PY{p}{)}\PY{p}{,}
    \PY{l+s+s2}{\PYZdq{}}\PY{l+s+s2}{Department\PYZus{}Name}\PY{l+s+s2}{\PYZdq{}}\PY{p}{:} \PY{n}{department\PYZus{}names}
\PY{p}{\PYZcb{}}
\PY{n}{department\PYZus{}df} \PY{o}{=} \PY{n}{pd}\PY{o}{.}\PY{n}{DataFrame}\PY{p}{(}\PY{n}{department\PYZus{}data}\PY{p}{)}

\PY{c+c1}{\PYZsh{} 5. ENROLLMENT\PYZus{}FACTS \PYZhy{} Generierung der Einschreibungen}
\PY{n}{enrollment\PYZus{}data} \PY{o}{=} \PY{p}{\PYZob{}}
    \PY{l+s+s2}{\PYZdq{}}\PY{l+s+s2}{Enrollment\PYZus{}ID}\PY{l+s+s2}{\PYZdq{}}\PY{p}{:} \PY{n+nb}{range}\PY{p}{(}\PY{l+m+mi}{1}\PY{p}{,} \PY{n}{num\PYZus{}enrollments} \PY{o}{+} \PY{l+m+mi}{1}\PY{p}{)}\PY{p}{,}
    \PY{l+s+s2}{\PYZdq{}}\PY{l+s+s2}{Student\PYZus{}ID}\PY{l+s+s2}{\PYZdq{}}\PY{p}{:} \PY{p}{[}\PY{n}{random}\PY{o}{.}\PY{n}{randint}\PY{p}{(}\PY{l+m+mi}{1}\PY{p}{,} \PY{n}{num\PYZus{}students}\PY{p}{)} \PY{k}{for} \PY{n}{\PYZus{}} \PY{o+ow}{in} \PY{n+nb}{range}\PY{p}{(}\PY{n}{num\PYZus{}enrollments}\PY{p}{)}\PY{p}{]}\PY{p}{,}
    \PY{l+s+s2}{\PYZdq{}}\PY{l+s+s2}{Course\PYZus{}ID}\PY{l+s+s2}{\PYZdq{}}\PY{p}{:} \PY{p}{[}\PY{n}{random}\PY{o}{.}\PY{n}{randint}\PY{p}{(}\PY{l+m+mi}{1}\PY{p}{,} \PY{n}{num\PYZus{}courses}\PY{p}{)} \PY{k}{for} \PY{n}{\PYZus{}} \PY{o+ow}{in} \PY{n+nb}{range}\PY{p}{(}\PY{n}{num\PYZus{}enrollments}\PY{p}{)}\PY{p}{]}\PY{p}{,}
    \PY{l+s+s2}{\PYZdq{}}\PY{l+s+s2}{Professor\PYZus{}ID}\PY{l+s+s2}{\PYZdq{}}\PY{p}{:} \PY{p}{[}\PY{n}{random}\PY{o}{.}\PY{n}{randint}\PY{p}{(}\PY{l+m+mi}{1}\PY{p}{,} \PY{n}{num\PYZus{}professors}\PY{p}{)} \PY{k}{for} \PY{n}{\PYZus{}} \PY{o+ow}{in} \PY{n+nb}{range}\PY{p}{(}\PY{n}{num\PYZus{}enrollments}\PY{p}{)}\PY{p}{]}\PY{p}{,}
    \PY{l+s+s2}{\PYZdq{}}\PY{l+s+s2}{Enrollment\PYZus{}Date}\PY{l+s+s2}{\PYZdq{}}\PY{p}{:} \PY{p}{[}\PY{n}{random\PYZus{}date}\PY{p}{(}\PY{n}{datetime}\PY{p}{(}\PY{l+m+mi}{2021}\PY{p}{,} \PY{l+m+mi}{1}\PY{p}{,} \PY{l+m+mi}{1}\PY{p}{)}\PY{p}{,} \PY{n}{datetime}\PY{p}{(}\PY{l+m+mi}{2023}\PY{p}{,} \PY{l+m+mi}{12}\PY{p}{,} \PY{l+m+mi}{31}\PY{p}{)}\PY{p}{)}\PY{o}{.}\PY{n}{date}\PY{p}{(}\PY{p}{)} \PY{k}{for} \PY{n}{\PYZus{}} \PY{o+ow}{in} \PY{n+nb}{range}\PY{p}{(}\PY{n}{num\PYZus{}enrollments}\PY{p}{)}\PY{p}{]}\PY{p}{,}
    \PY{l+s+s2}{\PYZdq{}}\PY{l+s+s2}{Grade}\PY{l+s+s2}{\PYZdq{}}\PY{p}{:} \PY{p}{[}\PY{n}{random}\PY{o}{.}\PY{n}{choice}\PY{p}{(}\PY{p}{[}\PY{l+s+s1}{\PYZsq{}}\PY{l+s+s1}{A}\PY{l+s+s1}{\PYZsq{}}\PY{p}{,} \PY{l+s+s1}{\PYZsq{}}\PY{l+s+s1}{B}\PY{l+s+s1}{\PYZsq{}}\PY{p}{,} \PY{l+s+s1}{\PYZsq{}}\PY{l+s+s1}{C}\PY{l+s+s1}{\PYZsq{}}\PY{p}{,} \PY{l+s+s1}{\PYZsq{}}\PY{l+s+s1}{D}\PY{l+s+s1}{\PYZsq{}}\PY{p}{,} \PY{l+s+s1}{\PYZsq{}}\PY{l+s+s1}{F}\PY{l+s+s1}{\PYZsq{}}\PY{p}{]}\PY{p}{)} \PY{k}{for} \PY{n}{\PYZus{}} \PY{o+ow}{in} \PY{n+nb}{range}\PY{p}{(}\PY{n}{num\PYZus{}enrollments}\PY{p}{)}\PY{p}{]}
\PY{p}{\PYZcb{}}
\PY{n}{enrollment\PYZus{}df} \PY{o}{=} \PY{n}{pd}\PY{o}{.}\PY{n}{DataFrame}\PY{p}{(}\PY{n}{enrollment\PYZus{}data}\PY{p}{)}

\PY{c+c1}{\PYZsh{} Speichern der DataFrames als CSV\PYZhy{}Dateien im \PYZsq{}tables\PYZsq{}\PYZhy{}Ordner}
\PY{n}{student\PYZus{}df}\PY{o}{.}\PY{n}{to\PYZus{}csv}\PY{p}{(}\PY{n}{os}\PY{o}{.}\PY{n}{path}\PY{o}{.}\PY{n}{join}\PY{p}{(}\PY{n}{output\PYZus{}folder}\PY{p}{,} \PY{l+s+s2}{\PYZdq{}}\PY{l+s+s2}{STUDENT\PYZus{}DIMENSION.csv}\PY{l+s+s2}{\PYZdq{}}\PY{p}{)}\PY{p}{,} \PY{n}{index}\PY{o}{=}\PY{k+kc}{False}\PY{p}{)}
\PY{n}{course\PYZus{}df}\PY{o}{.}\PY{n}{to\PYZus{}csv}\PY{p}{(}\PY{n}{os}\PY{o}{.}\PY{n}{path}\PY{o}{.}\PY{n}{join}\PY{p}{(}\PY{n}{output\PYZus{}folder}\PY{p}{,} \PY{l+s+s2}{\PYZdq{}}\PY{l+s+s2}{COURSE\PYZus{}DIMENSION.csv}\PY{l+s+s2}{\PYZdq{}}\PY{p}{)}\PY{p}{,} \PY{n}{index}\PY{o}{=}\PY{k+kc}{False}\PY{p}{)}
\PY{n}{professor\PYZus{}df}\PY{o}{.}\PY{n}{to\PYZus{}csv}\PY{p}{(}\PY{n}{os}\PY{o}{.}\PY{n}{path}\PY{o}{.}\PY{n}{join}\PY{p}{(}\PY{n}{output\PYZus{}folder}\PY{p}{,} \PY{l+s+s2}{\PYZdq{}}\PY{l+s+s2}{PROFESSOR\PYZus{}DIMENSION.csv}\PY{l+s+s2}{\PYZdq{}}\PY{p}{)}\PY{p}{,} \PY{n}{index}\PY{o}{=}\PY{k+kc}{False}\PY{p}{)}
\PY{n}{department\PYZus{}df}\PY{o}{.}\PY{n}{to\PYZus{}csv}\PY{p}{(}\PY{n}{os}\PY{o}{.}\PY{n}{path}\PY{o}{.}\PY{n}{join}\PY{p}{(}\PY{n}{output\PYZus{}folder}\PY{p}{,} \PY{l+s+s2}{\PYZdq{}}\PY{l+s+s2}{DEPARTMENT\PYZus{}DIMENSION.csv}\PY{l+s+s2}{\PYZdq{}}\PY{p}{)}\PY{p}{,} \PY{n}{index}\PY{o}{=}\PY{k+kc}{False}\PY{p}{)}
\PY{n}{enrollment\PYZus{}df}\PY{o}{.}\PY{n}{to\PYZus{}csv}\PY{p}{(}\PY{n}{os}\PY{o}{.}\PY{n}{path}\PY{o}{.}\PY{n}{join}\PY{p}{(}\PY{n}{output\PYZus{}folder}\PY{p}{,} \PY{l+s+s2}{\PYZdq{}}\PY{l+s+s2}{ENROLLMENT\PYZus{}FACTS.csv}\PY{l+s+s2}{\PYZdq{}}\PY{p}{)}\PY{p}{,} \PY{n}{index}\PY{o}{=}\PY{k+kc}{False}\PY{p}{)}

\PY{n+nb}{print}\PY{p}{(}\PY{l+s+s2}{\PYZdq{}}\PY{l+s+s2}{CSV\PYZhy{}Dateien wurden erfolgreich im }\PY{l+s+s2}{\PYZsq{}}\PY{l+s+s2}{tables}\PY{l+s+s2}{\PYZsq{}}\PY{l+s+s2}{\PYZhy{}Ordner gespeichert.}\PY{l+s+s2}{\PYZdq{}}\PY{p}{)}
\end{Verbatim}
\end{tcolorbox}

    \begin{Verbatim}[commandchars=\\\{\}]
CSV-Dateien wurden erfolgreich im 'tables'-Ordner gespeichert.
    \end{Verbatim}

    \begin{tcolorbox}[breakable, size=fbox, boxrule=1pt, pad at break*=1mm,colback=cellbackground, colframe=cellborder]
\prompt{In}{incolor}{81}{\boxspacing}
\begin{Verbatim}[commandchars=\\\{\}]
\PY{k+kn}{import} \PY{n+nn}{subprocess}

\PY{c+c1}{\PYZsh{} List available models in Ollama}
\PY{n}{process} \PY{o}{=} \PY{n}{subprocess}\PY{o}{.}\PY{n}{Popen}\PY{p}{(}\PY{p}{[}\PY{l+s+s2}{\PYZdq{}}\PY{l+s+s2}{ollama}\PY{l+s+s2}{\PYZdq{}}\PY{p}{,} \PY{l+s+s2}{\PYZdq{}}\PY{l+s+s2}{list}\PY{l+s+s2}{\PYZdq{}}\PY{p}{]}\PY{p}{,} \PY{n}{stdout}\PY{o}{=}\PY{n}{subprocess}\PY{o}{.}\PY{n}{PIPE}\PY{p}{,} \PY{n}{stderr}\PY{o}{=}\PY{n}{subprocess}\PY{o}{.}\PY{n}{PIPE}\PY{p}{)}

\PY{c+c1}{\PYZsh{} Ausgabe anzeigen}
\PY{n}{stdout}\PY{p}{,} \PY{n}{stderr} \PY{o}{=} \PY{n}{process}\PY{o}{.}\PY{n}{communicate}\PY{p}{(}\PY{p}{)}
\PY{n+nb}{print}\PY{p}{(}\PY{n}{stdout}\PY{o}{.}\PY{n}{decode}\PY{p}{(}\PY{l+s+s2}{\PYZdq{}}\PY{l+s+s2}{utf\PYZhy{}8}\PY{l+s+s2}{\PYZdq{}}\PY{p}{)}\PY{p}{)}

\PY{k}{if} \PY{n}{stderr}\PY{p}{:}
    \PY{n+nb}{print}\PY{p}{(}\PY{l+s+s2}{\PYZdq{}}\PY{l+s+s2}{Error:}\PY{l+s+s2}{\PYZdq{}}\PY{p}{,} \PY{n}{stderr}\PY{o}{.}\PY{n}{decode}\PY{p}{(}\PY{l+s+s2}{\PYZdq{}}\PY{l+s+s2}{utf\PYZhy{}8}\PY{l+s+s2}{\PYZdq{}}\PY{p}{)}\PY{p}{)}
\PY{k}{else}\PY{p}{:}
    \PY{n+nb}{print}\PY{p}{(}\PY{l+s+s2}{\PYZdq{}}\PY{l+s+s2}{Available models listed successfully.}\PY{l+s+s2}{\PYZdq{}}\PY{p}{)}
\end{Verbatim}
\end{tcolorbox}

    \begin{Verbatim}[commandchars=\\\{\}]
NAME                                ID              SIZE      MODIFIED
qwen2.5:7b                          845dbda0ea48    4.7 GB    Less than a second
ago
custom\_qwen2.5:7b\_seed\_42\_temp\_0    2849624f7fb4    4.7 GB    Less than a second
ago

Available models listed successfully.
    \end{Verbatim}

    \begin{tcolorbox}[breakable, size=fbox, boxrule=1pt, pad at break*=1mm,colback=cellbackground, colframe=cellborder]
\prompt{In}{incolor}{82}{\boxspacing}
\begin{Verbatim}[commandchars=\\\{\}]
\PY{k+kn}{from} \PY{n+nn}{src}\PY{n+nn}{.}\PY{n+nn}{variables} \PY{k+kn}{import} \PY{n}{database\PYZus{}schema\PYZus{}mermaid}
\end{Verbatim}
\end{tcolorbox}

    \begin{tcolorbox}[breakable, size=fbox, boxrule=1pt, pad at break*=1mm,colback=cellbackground, colframe=cellborder]
\prompt{In}{incolor}{83}{\boxspacing}
\begin{Verbatim}[commandchars=\\\{\}]
\PY{n}{preprompt} \PY{o}{=} \PY{l+s+s2}{\PYZdq{}}\PY{l+s+s2}{Your task is to generate executable SQL queries based on my provided model. Please adhere strictly to my model. For each of my questions, generate an executable SELECT SQL query. Do not provide any comments or explanations, only the query. Give me only one query per Question, make sure to start the query with }\PY{l+s+s2}{\PYZsq{}}\PY{l+s+s2}{SELECT}\PY{l+s+s2}{\PYZsq{}}\PY{l+s+s2}{ and end it with a }\PY{l+s+s2}{\PYZsq{}}\PY{l+s+s2}{;}\PY{l+s+s2}{\PYZsq{}}\PY{l+s+s2}{. I am using a sqllite\PYZhy{}Database.}\PY{l+s+s2}{\PYZdq{}}
\end{Verbatim}
\end{tcolorbox}

    \begin{tcolorbox}[breakable, size=fbox, boxrule=1pt, pad at break*=1mm,colback=cellbackground, colframe=cellborder]
\prompt{In}{incolor}{97}{\boxspacing}
\begin{Verbatim}[commandchars=\\\{\}]
\PY{c+c1}{\PYZsh{} Deine 5 Fachfragen in einer Liste}
\PY{n}{liste\PYZus{}fachfragen} \PY{o}{=} \PY{p}{[}
    \PY{l+s+s2}{\PYZdq{}}\PY{l+s+s2}{Welche Kurse werden am häufigsten von Studenten wiederholt? Zeige die Kursnamen und die Anzahl der Wiederholungen.}\PY{l+s+s2}{\PYZdq{}}\PY{p}{,}
    \PY{l+s+s2}{\PYZdq{}}\PY{l+s+s2}{Welcher Kurs hat die meisten Teilnehmer? Zeige den Kursnamen und die Teilnehmerzahl.}\PY{l+s+s2}{\PYZdq{}}\PY{p}{,}
    \PY{l+s+s2}{\PYZdq{}}\PY{l+s+s2}{Wie viele Einschreibungen gab es pro Monat im letzten Jahr? Zeige die Monate und die Anzahl der Einschreibungen.}\PY{l+s+s2}{\PYZdq{}}\PY{p}{,}
    \PY{l+s+s2}{\PYZdq{}}\PY{l+s+s2}{Welche Studenten haben sich im letzten Jahr eingeschrieben? Zeige die Namen und Einschreibedatum.}\PY{l+s+s2}{\PYZdq{}}\PY{p}{,}
    \PY{l+s+s2}{\PYZdq{}}\PY{l+s+s2}{Wie viele verschiedene Fächer unterrichtet jede Abteilung? Zeige die Abteilungsnamen und Anzahl der Fächer.}\PY{l+s+s2}{\PYZdq{}}\PY{p}{,}
    \PY{l+s+s2}{\PYZdq{}}\PY{l+s+s2}{Welche Studenten haben Kurse aus mindestens drei verschiedenen Abteilungen belegt? Zeige die Namen der Studenten und die Anzahl der Abteilungen.}\PY{l+s+s2}{\PYZdq{}}
\PY{p}{]}
\end{Verbatim}
\end{tcolorbox}

    \begin{tcolorbox}[breakable, size=fbox, boxrule=1pt, pad at break*=1mm,colback=cellbackground, colframe=cellborder]
\prompt{In}{incolor}{98}{\boxspacing}
\begin{Verbatim}[commandchars=\\\{\}]
\PY{n}{liste\PYZus{}model\PYZus{}output} \PY{o}{=} \PY{p}{[}\PY{p}{]} \PY{c+c1}{\PYZsh{} Liste, in welche gleich die Ouputs des Modells gepeichert werden}

\PY{c+c1}{\PYZsh{} Liste mit Prompts für das Modell erstellen}
\PY{n}{prompts} \PY{o}{=} \PY{p}{[}\PY{l+s+sa}{f}\PY{l+s+s2}{\PYZdq{}}\PY{l+s+si}{\PYZob{}}\PY{n}{database\PYZus{}schema\PYZus{}mermaid}\PY{l+s+si}{\PYZcb{}}\PY{l+s+s2}{ }\PY{l+s+si}{\PYZob{}}\PY{n}{preprompt}\PY{l+s+si}{\PYZcb{}}\PY{l+s+s2}{ }\PY{l+s+si}{\PYZob{}}\PY{n}{frage}\PY{l+s+si}{\PYZcb{}}\PY{l+s+s2}{\PYZdq{}} \PY{k}{for} \PY{n}{frage} \PY{o+ow}{in} \PY{n}{liste\PYZus{}fachfragen}\PY{p}{]}

\PY{c+c1}{\PYZsh{} Liste zum Speichern der Model Outputs}
\PY{n}{liste\PYZus{}model\PYZus{}output} \PY{o}{=} \PY{p}{[}\PY{p}{]}

\PY{c+c1}{\PYZsh{} Schleife über die Prompts}
\PY{k}{for} \PY{n}{prompt}\PY{p}{,} \PY{n}{frage} \PY{o+ow}{in} \PY{n+nb}{zip}\PY{p}{(}\PY{n}{prompts}\PY{p}{,} \PY{n}{liste\PYZus{}fachfragen}\PY{p}{)}\PY{p}{:}
    \PY{c+c1}{\PYZsh{} Modell für jede Frage neu starten und Prompt senden}
    \PY{n}{process} \PY{o}{=} \PY{n}{subprocess}\PY{o}{.}\PY{n}{Popen}\PY{p}{(}
        \PY{p}{[}\PY{l+s+s2}{\PYZdq{}}\PY{l+s+s2}{ollama}\PY{l+s+s2}{\PYZdq{}}\PY{p}{,} \PY{l+s+s2}{\PYZdq{}}\PY{l+s+s2}{run}\PY{l+s+s2}{\PYZdq{}}\PY{p}{,} \PY{n}{model\PYZus{}custom}\PY{p}{]}\PY{p}{,}
        \PY{n}{stdin}\PY{o}{=}\PY{n}{subprocess}\PY{o}{.}\PY{n}{PIPE}\PY{p}{,}
        \PY{n}{stdout}\PY{o}{=}\PY{n}{subprocess}\PY{o}{.}\PY{n}{PIPE}\PY{p}{,}
        \PY{n}{stderr}\PY{o}{=}\PY{n}{subprocess}\PY{o}{.}\PY{n}{PIPE}
    \PY{p}{)}
    
    \PY{c+c1}{\PYZsh{} Sende den Prompt an das Modell und empfange die Antwort}
    \PY{n}{stdout}\PY{p}{,} \PY{n}{stderr} \PY{o}{=} \PY{n}{process}\PY{o}{.}\PY{n}{communicate}\PY{p}{(}\PY{n+nb}{input}\PY{o}{=}\PY{n}{prompt}\PY{o}{.}\PY{n}{encode}\PY{p}{(}\PY{p}{)}\PY{p}{)}
    
    \PY{c+c1}{\PYZsh{} Output dekodieren und in der Liste speichern}
    \PY{n}{output} \PY{o}{=} \PY{n}{stdout}\PY{o}{.}\PY{n}{decode}\PY{p}{(}\PY{p}{)}\PY{o}{.}\PY{n}{strip}\PY{p}{(}\PY{p}{)}
    \PY{n}{liste\PYZus{}model\PYZus{}output}\PY{o}{.}\PY{n}{append}\PY{p}{(}\PY{n}{output}\PY{p}{)}  \PY{c+c1}{\PYZsh{} Speichern des Outputs in der Liste}
    
    \PY{c+c1}{\PYZsh{} Test für den Output:}
    \PY{c+c1}{\PYZsh{} print(f\PYZdq{}Antwort für die Frage \PYZsq{}\PYZob{}frage\PYZcb{}\PYZsq{}:\PYZbs{}n\PYZob{}output\PYZcb{}\PYZbs{}n\PYZdq{})}
    
    \PY{c+c1}{\PYZsh{} \PYZsh{} Optional: Fehlerbehandlung}
    \PY{c+c1}{\PYZsh{} if stderr:}
    \PY{c+c1}{\PYZsh{}     print(f\PYZdq{}Fehler bei der Verarbeitung der Frage \PYZsq{}\PYZob{}frage\PYZcb{}\PYZsq{}: \PYZob{}stderr.decode().strip()\PYZcb{}\PYZbs{}n\PYZdq{})}

\PY{c+c1}{\PYZsh{} Nach der Schleife: Die Liste liste\PYZus{}model\PYZus{}output enthält alle Antworten des Modells}
\end{Verbatim}
\end{tcolorbox}

    \begin{tcolorbox}[breakable, size=fbox, boxrule=1pt, pad at break*=1mm,colback=cellbackground, colframe=cellborder]
\prompt{In}{incolor}{99}{\boxspacing}
\begin{Verbatim}[commandchars=\\\{\}]
\PY{k+kn}{from} \PY{n+nn}{src}\PY{n+nn}{.}\PY{n+nn}{functions} \PY{k+kn}{import} \PY{n}{extract\PYZus{}sql\PYZus{}statement}
\end{Verbatim}
\end{tcolorbox}

    \begin{tcolorbox}[breakable, size=fbox, boxrule=1pt, pad at break*=1mm,colback=cellbackground, colframe=cellborder]
\prompt{In}{incolor}{100}{\boxspacing}
\begin{Verbatim}[commandchars=\\\{\}]
\PY{c+c1}{\PYZsh{} Funktion, die den Output des Modells so trimmt, dass nur das SQL\PYZhy{}Statement stehen bleibt.}
\PY{k+kn}{from} \PY{n+nn}{src}\PY{n+nn}{.}\PY{n+nn}{functions} \PY{k+kn}{import} \PY{n}{extract\PYZus{}sql\PYZus{}statement}

\PY{c+c1}{\PYZsh{} Beispielaufruf}
\PY{n}{output} \PY{o}{=} \PY{n}{stdout}\PY{o}{.}\PY{n}{decode}\PY{p}{(}\PY{p}{)}

\PY{c+c1}{\PYZsh{} Anwenden der Extraktionsfunktion}
\PY{n}{trimmed\PYZus{}output} \PY{o}{=} \PY{n}{extract\PYZus{}sql\PYZus{}statement}\PY{p}{(}\PY{n}{output}\PY{p}{)}

\PY{c+c1}{\PYZsh{} Test für die Ausgabe:}
\PY{c+c1}{\PYZsh{} print(trimmed\PYZus{}output)}
\end{Verbatim}
\end{tcolorbox}

    \begin{tcolorbox}[breakable, size=fbox, boxrule=1pt, pad at break*=1mm,colback=cellbackground, colframe=cellborder]
\prompt{In}{incolor}{101}{\boxspacing}
\begin{Verbatim}[commandchars=\\\{\}]
\PY{c+c1}{\PYZsh{} Importiere die Funktion, die das SQL\PYZhy{}Statement extrahiert}
\PY{k+kn}{from} \PY{n+nn}{src}\PY{n+nn}{.}\PY{n+nn}{functions} \PY{k+kn}{import} \PY{n}{extract\PYZus{}sql\PYZus{}statement}

\PY{c+c1}{\PYZsh{} Neue Liste, die nur die getrimmten SQL\PYZhy{}Statements speichern wird}
\PY{n}{liste\PYZus{}trimmed\PYZus{}sql} \PY{o}{=} \PY{p}{[}\PY{n}{extract\PYZus{}sql\PYZus{}statement}\PY{p}{(}\PY{n}{output}\PY{p}{)} \PY{k}{for} \PY{n}{output} \PY{o+ow}{in} \PY{n}{liste\PYZus{}model\PYZus{}output}\PY{p}{]}

\PY{c+c1}{\PYZsh{} \PYZsh{} Test: Ausgabe der getrimmten SQL\PYZhy{}Statements zur Kontrolle}
\PY{c+c1}{\PYZsh{} for trimmed\PYZus{}output in liste\PYZus{}trimmed\PYZus{}sql:}
\PY{c+c1}{\PYZsh{}     print(f\PYZsq{}\PYZob{}trimmed\PYZus{}output\PYZcb{}\PYZbs{}n\PYZsq{})}
\end{Verbatim}
\end{tcolorbox}

    \begin{tcolorbox}[breakable, size=fbox, boxrule=1pt, pad at break*=1mm,colback=cellbackground, colframe=cellborder]
\prompt{In}{incolor}{102}{\boxspacing}
\begin{Verbatim}[commandchars=\\\{\}]
\PY{k+kn}{import} \PY{n+nn}{sqlite3}
\PY{k+kn}{import} \PY{n+nn}{pandas} \PY{k}{as} \PY{n+nn}{pd}
\end{Verbatim}
\end{tcolorbox}

    \begin{tcolorbox}[breakable, size=fbox, boxrule=1pt, pad at break*=1mm,colback=cellbackground, colframe=cellborder]
\prompt{In}{incolor}{103}{\boxspacing}
\begin{Verbatim}[commandchars=\\\{\}]
\PY{k+kn}{from} \PY{n+nn}{src}\PY{n+nn}{.}\PY{n+nn}{functions} \PY{k+kn}{import} \PY{n}{create\PYZus{}tables}

\PY{c+c1}{\PYZsh{} Datenbank und Tabellen erstellen}
\PY{n}{create\PYZus{}tables}\PY{p}{(}\PY{l+s+s1}{\PYZsq{}}\PY{l+s+s1}{university\PYZus{}database.db}\PY{l+s+s1}{\PYZsq{}}\PY{p}{)}
\end{Verbatim}
\end{tcolorbox}

    \begin{Verbatim}[commandchars=\\\{\}]
Tabellen wurden erfolgreich erstellt.
    \end{Verbatim}

    \begin{tcolorbox}[breakable, size=fbox, boxrule=1pt, pad at break*=1mm,colback=cellbackground, colframe=cellborder]
\prompt{In}{incolor}{104}{\boxspacing}
\begin{Verbatim}[commandchars=\\\{\}]
\PY{c+c1}{\PYZsh{} Verbindung und Cursor erstellen}
\PY{n}{conn} \PY{o}{=} \PY{n}{sqlite3}\PY{o}{.}\PY{n}{connect}\PY{p}{(}\PY{l+s+s1}{\PYZsq{}}\PY{l+s+s1}{university\PYZus{}database.db}\PY{l+s+s1}{\PYZsq{}}\PY{p}{)}
\PY{n}{cursor} \PY{o}{=} \PY{n}{conn}\PY{o}{.}\PY{n}{cursor}\PY{p}{(}\PY{p}{)}
\end{Verbatim}
\end{tcolorbox}

    \begin{tcolorbox}[breakable, size=fbox, boxrule=1pt, pad at break*=1mm,colback=cellbackground, colframe=cellborder]
\prompt{In}{incolor}{105}{\boxspacing}
\begin{Verbatim}[commandchars=\\\{\}]
\PY{c+c1}{\PYZsh{} Zeige alle Tabellen in der Datenbank an}
\PY{n}{cursor}\PY{o}{.}\PY{n}{execute}\PY{p}{(}\PY{l+s+s2}{\PYZdq{}}\PY{l+s+s2}{SELECT name FROM sqlite\PYZus{}master WHERE type=}\PY{l+s+s2}{\PYZsq{}}\PY{l+s+s2}{table}\PY{l+s+s2}{\PYZsq{}}\PY{l+s+s2}{;}\PY{l+s+s2}{\PYZdq{}}\PY{p}{)}
\PY{n+nb}{print}\PY{p}{(}\PY{n}{cursor}\PY{o}{.}\PY{n}{fetchall}\PY{p}{(}\PY{p}{)}\PY{p}{)}
\end{Verbatim}
\end{tcolorbox}

    \begin{Verbatim}[commandchars=\\\{\}]
[('COURSE\_DIMENSION',), ('DEPARTMENT\_DIMENSION',), ('ENROLLMENT\_FACTS',),
('PROFESSOR\_DIMENSION',), ('STUDENT\_DIMENSION',)]
    \end{Verbatim}

    \begin{tcolorbox}[breakable, size=fbox, boxrule=1pt, pad at break*=1mm,colback=cellbackground, colframe=cellborder]
\prompt{In}{incolor}{106}{\boxspacing}
\begin{Verbatim}[commandchars=\\\{\}]
\PY{k+kn}{from} \PY{n+nn}{src}\PY{n+nn}{.}\PY{n+nn}{functions} \PY{k+kn}{import} \PY{n}{import\PYZus{}csv\PYZus{}to\PYZus{}sqlite}
\end{Verbatim}
\end{tcolorbox}

    \begin{tcolorbox}[breakable, size=fbox, boxrule=1pt, pad at break*=1mm,colback=cellbackground, colframe=cellborder]
\prompt{In}{incolor}{107}{\boxspacing}
\begin{Verbatim}[commandchars=\\\{\}]
\PY{k+kn}{from} \PY{n+nn}{src}\PY{n+nn}{.}\PY{n+nn}{functions} \PY{k+kn}{import} \PY{n}{import\PYZus{}all\PYZus{}csv\PYZus{}to\PYZus{}sqlite}
\end{Verbatim}
\end{tcolorbox}

    \begin{tcolorbox}[breakable, size=fbox, boxrule=1pt, pad at break*=1mm,colback=cellbackground, colframe=cellborder]
\prompt{In}{incolor}{108}{\boxspacing}
\begin{Verbatim}[commandchars=\\\{\}]
\PY{c+c1}{\PYZsh{} Beispielaufruf für den Ordner \PYZdq{}tables\PYZdq{} und eine persistente Datenbank}
\PY{n}{import\PYZus{}all\PYZus{}csv\PYZus{}to\PYZus{}sqlite}\PY{p}{(}\PY{l+s+s1}{\PYZsq{}}\PY{l+s+s1}{tables}\PY{l+s+s1}{\PYZsq{}}\PY{p}{,} \PY{n}{db\PYZus{}name}\PY{o}{=}\PY{l+s+s1}{\PYZsq{}}\PY{l+s+s1}{university\PYZus{}database.db}\PY{l+s+s1}{\PYZsq{}}\PY{p}{)}
\end{Verbatim}
\end{tcolorbox}

    \begin{Verbatim}[commandchars=\\\{\}]
Importiert: COURSE\_DIMENSION
Tabelle 'COURSE\_DIMENSION' wurde aus 'COURSE\_DIMENSION.csv' importiert.
Importiert: DEPARTMENT\_DIMENSION
Tabelle 'DEPARTMENT\_DIMENSION' wurde aus 'DEPARTMENT\_DIMENSION.csv' importiert.
Importiert: ENROLLMENT\_FACTS
Tabelle 'ENROLLMENT\_FACTS' wurde aus 'ENROLLMENT\_FACTS.csv' importiert.
    \end{Verbatim}

    \begin{Verbatim}[commandchars=\\\{\}]
Importiert: PROFESSOR\_DIMENSION
Tabelle 'PROFESSOR\_DIMENSION' wurde aus 'PROFESSOR\_DIMENSION.csv' importiert.
Importiert: STUDENT\_DIMENSION
Tabelle 'STUDENT\_DIMENSION' wurde aus 'STUDENT\_DIMENSION.csv' importiert.
    \end{Verbatim}

    \begin{tcolorbox}[breakable, size=fbox, boxrule=1pt, pad at break*=1mm,colback=cellbackground, colframe=cellborder]
\prompt{In}{incolor}{109}{\boxspacing}
\begin{Verbatim}[commandchars=\\\{\}]
\PY{k+kn}{import} \PY{n+nn}{pandas} \PY{k}{as} \PY{n+nn}{pd}
\PY{k+kn}{from} \PY{n+nn}{IPython}\PY{n+nn}{.}\PY{n+nn}{display} \PY{k+kn}{import} \PY{n}{display}\PY{p}{,} \PY{n}{Markdown}

\PY{c+c1}{\PYZsh{} Schleife über die Fachfragen und zugehörigen SQL\PYZhy{}Statements}
\PY{k}{for} \PY{n}{frage}\PY{p}{,} \PY{n}{sql\PYZus{}query} \PY{o+ow}{in} \PY{n+nb}{zip}\PY{p}{(}\PY{n}{liste\PYZus{}fachfragen}\PY{p}{,} \PY{n}{liste\PYZus{}trimmed\PYZus{}sql}\PY{p}{)}\PY{p}{:}
    \PY{k}{try}\PY{p}{:}
        \PY{c+c1}{\PYZsh{} Führe die SQL\PYZhy{}Abfrage aus}
        \PY{n}{cursor}\PY{o}{.}\PY{n}{execute}\PY{p}{(}\PY{n}{sql\PYZus{}query}\PY{p}{)}
        
        \PY{c+c1}{\PYZsh{} Markdown zur Frage anzeigen}
        \PY{n}{display}\PY{p}{(}\PY{n}{Markdown}\PY{p}{(}\PY{l+s+sa}{f}\PY{l+s+s2}{\PYZdq{}}\PY{l+s+s2}{\PYZsh{}\PYZsh{}\PYZsh{} An das Modell gestellte Frage:}\PY{l+s+se}{\PYZbs{}n}\PY{l+s+si}{\PYZob{}}\PY{n}{frage}\PY{l+s+si}{\PYZcb{}}\PY{l+s+se}{\PYZbs{}n}\PY{l+s+s2}{\PYZdq{}}\PY{p}{)}\PY{p}{)}
        
        \PY{c+c1}{\PYZsh{} SQL\PYZhy{}Abfrage anzeigen}
        \PY{n}{display}\PY{p}{(}\PY{n}{Markdown}\PY{p}{(}\PY{l+s+sa}{f}\PY{l+s+s2}{\PYZdq{}}\PY{l+s+s2}{\PYZsh{}\PYZsh{}\PYZsh{}\PYZsh{} Durch das Modell generierte SQL\PYZhy{}Abfrage:}\PY{l+s+se}{\PYZbs{}n}\PY{l+s+se}{\PYZbs{}n}\PY{l+s+s2}{```sql}\PY{l+s+se}{\PYZbs{}n}\PY{l+s+si}{\PYZob{}}\PY{n}{sql\PYZus{}query}\PY{l+s+si}{\PYZcb{}}\PY{l+s+se}{\PYZbs{}n}\PY{l+s+s2}{```}\PY{l+s+se}{\PYZbs{}n}\PY{l+s+s2}{\PYZdq{}}\PY{p}{)}\PY{p}{)}
        
        \PY{c+c1}{\PYZsh{} Ergebnisse der SQL\PYZhy{}Abfrage in ein DataFrame laden und anzeigen}
        \PY{n}{df} \PY{o}{=} \PY{n}{pd}\PY{o}{.}\PY{n}{DataFrame}\PY{p}{(}\PY{n}{cursor}\PY{o}{.}\PY{n}{fetchall}\PY{p}{(}\PY{p}{)}\PY{p}{,} \PY{n}{columns}\PY{o}{=}\PY{p}{[}\PY{n}{description}\PY{p}{[}\PY{l+m+mi}{0}\PY{p}{]} \PY{k}{for} \PY{n}{description} \PY{o+ow}{in} \PY{n}{cursor}\PY{o}{.}\PY{n}{description}\PY{p}{]}\PY{p}{)}
        
        \PY{c+c1}{\PYZsh{} DataFrame anzeigen}
        \PY{n}{display}\PY{p}{(}\PY{n}{Markdown}\PY{p}{(}\PY{l+s+s2}{\PYZdq{}}\PY{l+s+s2}{\PYZsh{}\PYZsh{}\PYZsh{}\PYZsh{} Ergebnis der durch das Modell generierten SQL\PYZhy{}Abfrage:}\PY{l+s+s2}{\PYZdq{}}\PY{p}{)}\PY{p}{)}
        \PY{n}{display}\PY{p}{(}\PY{n}{df}\PY{p}{)}
        
        \PY{c+c1}{\PYZsh{} Trennlinie zur besseren Übersicht}
        \PY{n}{display}\PY{p}{(}\PY{n}{Markdown}\PY{p}{(}\PY{l+s+s2}{\PYZdq{}}\PY{l+s+s2}{\PYZhy{}\PYZhy{}\PYZhy{}}\PY{l+s+s2}{\PYZdq{}}\PY{p}{)}\PY{p}{)}

    \PY{k}{except} \PY{n+ne}{Exception} \PY{k}{as} \PY{n}{e}\PY{p}{:}
        \PY{c+c1}{\PYZsh{} Fehlerbehandlung bei SQL\PYZhy{}Ausführungsfehlern mit Markdown}
        \PY{n}{display}\PY{p}{(}\PY{n}{Markdown}\PY{p}{(}\PY{l+s+sa}{f}\PY{l+s+s2}{\PYZdq{}}\PY{l+s+s2}{**Fehler bei der Ausführung der Abfrage für die Frage }\PY{l+s+s2}{\PYZsq{}}\PY{l+s+si}{\PYZob{}}\PY{n}{frage}\PY{l+s+si}{\PYZcb{}}\PY{l+s+s2}{\PYZsq{}}\PY{l+s+s2}{:** }\PY{l+s+si}{\PYZob{}}\PY{n}{e}\PY{l+s+si}{\PYZcb{}}\PY{l+s+se}{\PYZbs{}n}\PY{l+s+s2}{\PYZdq{}}\PY{p}{)}\PY{p}{)}
        \PY{n}{display}\PY{p}{(}\PY{n}{Markdown}\PY{p}{(}\PY{l+s+s2}{\PYZdq{}}\PY{l+s+s2}{\PYZhy{}\PYZhy{}\PYZhy{}}\PY{l+s+s2}{\PYZdq{}}\PY{p}{)}\PY{p}{)}  \PY{c+c1}{\PYZsh{} Trennlinie für Fehlerfall}
\end{Verbatim}
\end{tcolorbox}

    \hypertarget{an-das-modell-gestellte-frage}{%
\subsubsection{An das Modell gestellte
Frage:}\label{an-das-modell-gestellte-frage}}

Welche Kurse werden am häufigsten von Studenten wiederholt? Zeige die
Kursnamen und die Anzahl der Wiederholungen.

    
    \hypertarget{durch-das-modell-generierte-sql-abfrage}{%
\paragraph{Durch das Modell generierte
SQL-Abfrage:}\label{durch-das-modell-generierte-sql-abfrage}}

\begin{Shaded}
\begin{Highlighting}[]
\KeywordTok{SELECT}\NormalTok{ C.Course\_Name, }\FunctionTok{COUNT}\NormalTok{(E.Enrollment\_ID) }\KeywordTok{AS}\NormalTok{ Repeated\_Course\_Count}
\KeywordTok{FROM}\NormalTok{ ENROLLMENT\_FACTS E}
\KeywordTok{JOIN}\NormalTok{ COURSE\_DIMENSION C }\KeywordTok{ON}\NormalTok{ E.Course\_ID }\OperatorTok{=}\NormalTok{ C.Course\_ID}
\KeywordTok{WHERE}\NormalTok{ E.Grade }\KeywordTok{IN}\NormalTok{ (}\StringTok{\textquotesingle{}F\textquotesingle{}}\NormalTok{, }\StringTok{\textquotesingle{}W\textquotesingle{}}\NormalTok{)}
\KeywordTok{GROUP} \KeywordTok{BY}\NormalTok{ C.Course\_Name}
\KeywordTok{ORDER} \KeywordTok{BY}\NormalTok{ Repeated\_Course\_Count }\KeywordTok{DESC}\NormalTok{;}
\end{Highlighting}
\end{Shaded}

    
    \hypertarget{ergebnis-der-durch-das-modell-generierten-sql-abfrage}{%
\paragraph{Ergebnis der durch das Modell generierten
SQL-Abfrage:}\label{ergebnis-der-durch-das-modell-generierten-sql-abfrage}}

    
    
    \begin{Verbatim}[commandchars=\\\{\}]
                   Course\_Name  Repeated\_Course\_Count
0      Programming with Python                     11
1                Deep Learning                      9
2           Data Visualization                      8
3                 Ethics in AI                      7
4  Statistics for Data Science                      6
5            Big Data Analysis                      6
6             Database Systems                      4
7          Data Science Basics                      4
8    Advanced Machine Learning                      4
9           Project Management                      3
    \end{Verbatim}

    
    \begin{center}\rule{0.5\linewidth}{0.5pt}\end{center}

    
    \hypertarget{an-das-modell-gestellte-frage}{%
\subsubsection{An das Modell gestellte
Frage:}\label{an-das-modell-gestellte-frage}}

Welcher Kurs hat die meisten Teilnehmer? Zeige den Kursnamen und die
Teilnehmerzahl.

    
    \hypertarget{durch-das-modell-generierte-sql-abfrage}{%
\paragraph{Durch das Modell generierte
SQL-Abfrage:}\label{durch-das-modell-generierte-sql-abfrage}}

\begin{Shaded}
\begin{Highlighting}[]
\KeywordTok{SELECT}\NormalTok{ C.Course\_Name, }\FunctionTok{COUNT}\NormalTok{(E.Student\_ID) }\KeywordTok{AS}\NormalTok{ Participant\_Count}
\KeywordTok{FROM}\NormalTok{ ENROLLMENT\_FACTS E}
\KeywordTok{JOIN}\NormalTok{ COURSE\_DIMENSION C }\KeywordTok{ON}\NormalTok{ E.Course\_ID }\OperatorTok{=}\NormalTok{ C.Course\_ID}
\KeywordTok{GROUP} \KeywordTok{BY}\NormalTok{ C.Course\_Name}
\KeywordTok{ORDER} \KeywordTok{BY}\NormalTok{ Participant\_Count }\KeywordTok{DESC}
\KeywordTok{LIMIT} \DecValTok{1}\NormalTok{;}
\end{Highlighting}
\end{Shaded}

    
    \hypertarget{ergebnis-der-durch-das-modell-generierten-sql-abfrage}{%
\paragraph{Ergebnis der durch das Modell generierten
SQL-Abfrage:}\label{ergebnis-der-durch-das-modell-generierten-sql-abfrage}}

    
    
    \begin{Verbatim}[commandchars=\\\{\}]
         Course\_Name  Participant\_Count
0  Big Data Analysis                 40
    \end{Verbatim}

    
    \begin{center}\rule{0.5\linewidth}{0.5pt}\end{center}

    
    \hypertarget{an-das-modell-gestellte-frage}{%
\subsubsection{An das Modell gestellte
Frage:}\label{an-das-modell-gestellte-frage}}

Wie viele Einschreibungen gab es pro Monat im letzten Jahr? Zeige die
Monate und die Anzahl der Einschreibungen.

    
    \hypertarget{durch-das-modell-generierte-sql-abfrage}{%
\paragraph{Durch das Modell generierte
SQL-Abfrage:}\label{durch-das-modell-generierte-sql-abfrage}}

\begin{Shaded}
\begin{Highlighting}[]
\KeywordTok{SELECT}\NormalTok{ strftime(}\StringTok{\textquotesingle{}\%Y{-}\%m\textquotesingle{}}\NormalTok{, Enrollment\_Date) }\KeywordTok{AS} \DataTypeTok{Month}\NormalTok{, }\FunctionTok{COUNT}\NormalTok{(}\OperatorTok{*}\NormalTok{) }\KeywordTok{AS}\NormalTok{ Enrollment\_Count }
\KeywordTok{FROM}\NormalTok{ ENROLLMENT\_FACTS }
\KeywordTok{WHERE}\NormalTok{ Enrollment\_Date }\KeywordTok{BETWEEN} \DataTypeTok{DATE}\NormalTok{(}\StringTok{\textquotesingle{}now\textquotesingle{}}\NormalTok{, }\StringTok{\textquotesingle{}{-}1 year\textquotesingle{}}\NormalTok{) }\KeywordTok{AND} \DataTypeTok{DATE}\NormalTok{(}\StringTok{\textquotesingle{}now\textquotesingle{}}\NormalTok{) }
\KeywordTok{GROUP} \KeywordTok{BY} \DataTypeTok{Month}\NormalTok{;}
\end{Highlighting}
\end{Shaded}

    
    \hypertarget{ergebnis-der-durch-das-modell-generierten-sql-abfrage}{%
\paragraph{Ergebnis der durch das Modell generierten
SQL-Abfrage:}\label{ergebnis-der-durch-das-modell-generierten-sql-abfrage}}

    
    
    \begin{Verbatim}[commandchars=\\\{\}]
     Month  Enrollment\_Count
0  2023-11                 1
1  2023-12                 6
    \end{Verbatim}

    
    \begin{center}\rule{0.5\linewidth}{0.5pt}\end{center}

    
    \hypertarget{an-das-modell-gestellte-frage}{%
\subsubsection{An das Modell gestellte
Frage:}\label{an-das-modell-gestellte-frage}}

Welche Studenten haben sich im letzten Jahr eingeschrieben? Zeige die
Namen und Einschreibedatum.

    
    \hypertarget{durch-das-modell-generierte-sql-abfrage}{%
\paragraph{Durch das Modell generierte
SQL-Abfrage:}\label{durch-das-modell-generierte-sql-abfrage}}

\begin{Shaded}
\begin{Highlighting}[]
\KeywordTok{SELECT}\NormalTok{ S.First\_Name, S.Last\_Name, E.Enrollment\_Date }
\KeywordTok{FROM}\NormalTok{ STUDENT\_DIMENSION S }
\KeywordTok{JOIN}\NormalTok{ ENROLLMENT\_FACTS E }\KeywordTok{ON}\NormalTok{ S.Student\_ID }\OperatorTok{=}\NormalTok{ E.Student\_ID }
\KeywordTok{WHERE}\NormalTok{ E.Enrollment\_Date }\OperatorTok{\textgreater{}=} \DataTypeTok{DATE}\NormalTok{(}\StringTok{\textquotesingle{}now\textquotesingle{}}\NormalTok{, }\StringTok{\textquotesingle{}{-}1 year\textquotesingle{}}\NormalTok{);}
\end{Highlighting}
\end{Shaded}

    
    \hypertarget{ergebnis-der-durch-das-modell-generierten-sql-abfrage}{%
\paragraph{Ergebnis der durch das Modell generierten
SQL-Abfrage:}\label{ergebnis-der-durch-das-modell-generierten-sql-abfrage}}

    
    
    \begin{Verbatim}[commandchars=\\\{\}]
  First\_Name  Last\_Name Enrollment\_Date
0    Whitney    Hensley      2023-12-24
1   Patricia  Rodriguez      2023-12-24
2      Tasha       Kidd      2023-12-14
3     George     Morgan      2023-11-18
4      Debra     Gaines      2023-12-04
5    Jeffrey     Morgan      2023-12-23
6     Amanda   Humphrey      2023-12-23
    \end{Verbatim}

    
    \begin{center}\rule{0.5\linewidth}{0.5pt}\end{center}

    
    \hypertarget{an-das-modell-gestellte-frage}{%
\subsubsection{An das Modell gestellte
Frage:}\label{an-das-modell-gestellte-frage}}

Wie viele verschiedene Fächer unterrichtet jede Abteilung? Zeige die
Abteilungsnamen und Anzahl der Fächer.

    
    \hypertarget{durch-das-modell-generierte-sql-abfrage}{%
\paragraph{Durch das Modell generierte
SQL-Abfrage:}\label{durch-das-modell-generierte-sql-abfrage}}

\begin{Shaded}
\begin{Highlighting}[]
\KeywordTok{SELECT}\NormalTok{ DEPARTMENT\_DIMENSION.Department\_Name, }\FunctionTok{COUNT}\NormalTok{(COURSE\_DIMENSION.Course\_ID) }\KeywordTok{AS}\NormalTok{ Number\_of\_Courses}
\KeywordTok{FROM}\NormalTok{ COURSE\_DIMENSION}
\KeywordTok{JOIN}\NormalTok{ DEPARTMENT\_DIMENSION }\KeywordTok{ON}\NormalTok{ COURSE\_DIMENSION.Department\_ID }\OperatorTok{=}\NormalTok{ DEPARTMENT\_DIMENSION.Department\_ID}
\KeywordTok{GROUP} \KeywordTok{BY}\NormalTok{ DEPARTMENT\_DIMENSION.Department\_Name;}
\end{Highlighting}
\end{Shaded}

    
    \hypertarget{ergebnis-der-durch-das-modell-generierten-sql-abfrage}{%
\paragraph{Ergebnis der durch das Modell generierten
SQL-Abfrage:}\label{ergebnis-der-durch-das-modell-generierten-sql-abfrage}}

    
    
    \begin{Verbatim}[commandchars=\\\{\}]
           Department\_Name  Number\_of\_Courses
0  Business Administration                  3
1         Computer Science                  4
2               Psychology                  3
    \end{Verbatim}

    
    \begin{center}\rule{0.5\linewidth}{0.5pt}\end{center}

    
    \hypertarget{an-das-modell-gestellte-frage}{%
\subsubsection{An das Modell gestellte
Frage:}\label{an-das-modell-gestellte-frage}}

Welche Studenten haben Kurse aus mindestens drei verschiedenen
Abteilungen belegt? Zeige die Namen der Studenten und die Anzahl der
Abteilungen.

    
    \hypertarget{durch-das-modell-generierte-sql-abfrage}{%
\paragraph{Durch das Modell generierte
SQL-Abfrage:}\label{durch-das-modell-generierte-sql-abfrage}}

\begin{Shaded}
\begin{Highlighting}[]
\KeywordTok{SELECT}\NormalTok{ S.First\_Name, S.Last\_Name, }\FunctionTok{COUNT}\NormalTok{(}\KeywordTok{DISTINCT}\NormalTok{ D.Department\_Name) }\KeywordTok{AS}\NormalTok{ Number\_of\_Departments}
\KeywordTok{FROM}\NormalTok{ STUDENT\_DIMENSION S}
\KeywordTok{JOIN}\NormalTok{ ENROLLMENT\_FACTS E }\KeywordTok{ON}\NormalTok{ S.Student\_ID }\OperatorTok{=}\NormalTok{ E.Student\_ID}
\KeywordTok{JOIN}\NormalTok{ COURSE\_DIMENSION C }\KeywordTok{ON}\NormalTok{ E.Course\_ID }\OperatorTok{=}\NormalTok{ C.Course\_ID}
\KeywordTok{JOIN}\NormalTok{ DEPARTMENT\_DIMENSION D }\KeywordTok{ON}\NormalTok{ C.Department\_ID }\OperatorTok{=}\NormalTok{ D.Department\_ID}
\KeywordTok{GROUP} \KeywordTok{BY}\NormalTok{ S.Student\_ID, S.First\_Name, S.Last\_Name}
\KeywordTok{HAVING} \FunctionTok{COUNT}\NormalTok{(}\KeywordTok{DISTINCT}\NormalTok{ D.Department\_Name) }\OperatorTok{\textgreater{}=} \DecValTok{3}\NormalTok{;}
\end{Highlighting}
\end{Shaded}

    
    \hypertarget{ergebnis-der-durch-das-modell-generierten-sql-abfrage}{%
\paragraph{Ergebnis der durch das Modell generierten
SQL-Abfrage:}\label{ergebnis-der-durch-das-modell-generierten-sql-abfrage}}

    
    
    \begin{Verbatim}[commandchars=\\\{\}]
     First\_Name  Last\_Name  Number\_of\_Departments
0      Danielle      Smith                      3
1       Jeffrey    Herrera                      3
2       Anthony      Mason                      3
3         James      White                      3
4       Jeffrey      Evans                      3
5          Adam  Hernandez                      3
6         Linda     Farmer                      3
7         Susan      Smith                      3
8   Christopher      Smith                      3
9       Lindsay     Parker                      3
10      Jessica      Brown                      3
11       Teresa     Arroyo                      3
12       Jeremy      Baker                      3
13      Deborah   Martinez                      3
14      Chelsea     Walker                      3
15       Angela      Perez                      3
16      Leonard    Mueller                      3
17        Diana    Aguirre                      3
18      Rachael     Garcia                      3
19     Reginald     Powell                      3
20      Jeffrey     Morgan                      3
21      Whitney    Hensley                      3
22        Bryan      Ramos                      3
    \end{Verbatim}

    
    \begin{center}\rule{0.5\linewidth}{0.5pt}\end{center}

    

    % Add a bibliography block to the postdoc
    
    
    
\end{document}
